\PassOptionsToPackage{dvipsnames,table}{xcolor}
\documentclass[10pt]{beamer}
\usepackage{Cours}

\begin{document}

\input{\detokenize{/home/fenarius/Travail/Cours/NSIPremiere/docs/commun/MacrosCours.tex}}
\setcounter{numchap}{7}

\newcommand{\Arch}{\cnum Architecture des ordinateurs}


% Définition langage de programmation
\begin{frame}
	\mframe{\Arch}
	\begin{alertblock}{Modèle de Von Neumann}
		\begin{itemize}
			\item<1-> Les ordinateurs modernes sont construits autour d'un modèle défini par le mathématicien John Von Neumann en 1945 et appelé \textcolor{blue}{Architecture de Von Neumann}.
			\item<2-> Dans ce modèle, l'ordinateur se décompose en 5 parties distinctes :
			      \begin{enumerate}
				      \item<3-> Les dispositifs d'\textcolor{blue}{entrée} des données (ex : clavier, souris, écran tactile, réseau \dots),
				      \item<4-> La \textcolor{blue}{mémoire} qui stocke les données et les programmes (ex : mémoire cache, {\sc ram}, \dots)
				      \item<5-> L'\textcolor{blue}{unité arithmétique et logique {\sc ual}} qui effectue les opérations (addition, soustraction, comparaison, \dots) sur les données.
				      \item<6-> L'\textcolor{blue}{unité de contrôle} qui est chargé de la gestion de l'ordre des opérations (séquençage)
				      \item<7-> Les dispositifs de \textcolor{blue}{sortie} des données (ex : écran, imprimante, \dots)
			      \end{enumerate}
		\end{itemize}
	\end{alertblock}
\end{frame}

%Remarque architectecture von neumann
\begin{frame}
	\mframe{\Arch}
	\begin{block}{Remarques :}
		\begin{itemize}
			\item<1-> Dans les ordinateurs modernes, l'{\sc ual} et l'unité de contrôle sont regroupés dans le processeur ({\sc cpu} pour Central Processing Unit en anglais)
			\item<2-> Certains periphériques sont à la fois des dispositifs d'entrée et de sortie. Par exemple, le disque dur car on peut y lire (entrée) et écrire (sortie) des données.
			\item<3-> Par rapport au modèle initial, les ordinateurs actuels possèdent parfois plusieurs processeurs ou coeurs.
		\end{itemize}
	\end{block}
\end{frame}


%Schéma
\begin{frame}
	\mframe{\Arch}
	\setlength{\shadowsize}{1pt}
	\psset{linewidth=0.7pt}
	\begin{block}{Schéma représentant l'architecture de Von Neumann :}
		\begin{tabularx}{0.9\textwidth}{Xp{1cm}|Y|p{1cm}X}
			\cline{3-3}
			                                                                                                                                                                                                                                                                      &  & \textcolor{blue}{\sc cpu} &  & \\
			\onslide<3->{\rnode{ram}{\shadowbox{\makebox[2cm]{\par\noindent\rule[-1.4cm]{0pt}{3cm} \textcolor{red}{Mémoire} }}} }                                                                                                                                                 &  &
			\onslide<2->{\rnode{ual}{\shadowbox{\makebox[2cm]{\par\noindent\rule[-0.4cm]{0pt}{1cm} {\textcolor{red}{\sc ual}} }}} \newline \rule{0pt}{0.8cm} \newline {\rnode{uc}{\shadowbox{\makebox[2cm]{\par\noindent\rule[-0.4cm]{0pt}{1cm} \textcolor{red}{\sc uc} }}} }   } &  &
			\onslide<4->{\rnode{in}{\shadowbox{\makebox[2cm]{\par\noindent\rule[-0.4cm]{0pt}{1cm} {\textcolor{red}{ Entrées}} }}} \newline \rule{0pt}{0.8cm} {\rnode{out}{\shadowbox{\makebox[2cm]{\par\noindent\rule[-0.4cm]{0pt}{1cm} \textcolor{red}{ Sorties} }}} }   }                                           \\
			\cline{3-3}
			\onslide<5->{\ncline[offsetA=0.45cm, nodesepA=0.1cm, linecolor=blue]{->}{ram}{ual}}
			\onslide<5->{\ncline[offsetA=-0.3cm, offsetB=-0.75cm,nodesepB=0.15cm, nodesepA=-0.1cm, linecolor=blue]{->}{ual}{ram}}
			\onslide<6->{\ncline[offsetA=-0.48cm, nodesepA=0.1cm, linecolor=blue]{->}{ram}{uc}}
			\onslide<6->{\ncline[offsetA=0.3cm, offsetB=0.78cm,nodesepB=0.15cm, nodesepA=-0.1cm, linecolor=blue]{->}{uc}{ram}}
			\onslide<7->{\ncline[offset=0.2cm,linecolor=blue]{->}{ual}{uc}}
			\onslide<7->{\ncline[offset=0.2cm,linecolor=blue]{->}{uc}{ual}}
			\onslide<8->{\ncline[linecolor=blue]{->}{in}{ual}}
			\onslide<9->{\ncline[linecolor=blue,offsetA=0.2cm,offsetB=-0.2cm,nodesepA=-0.1cm,nodesepB=-0.1cm]{->}{ual}{out}}
		\end{tabularx}
	\end{block}
\end{frame}


% Transistor et booléens
\begin{frame}
	\mframe{\Arch}
	\begin{block}{Remarques :}
		\begin{itemize}
			\item<1-> Le composant de base des ordinateurs est le \textit{transistor}, un composant électronique ne pouvant être que dans deux états. Soit il laisse passer le courant (état \textcolor{red}{1}), soit il ne le laisse pas passer (état \textcolor{red}{0}).
			\item<2-> Toutes les données représentées dans un ordinateur le sont donc sous forme de 0 et de 1.
			\item<3-> Dès les années 1850, dans des travaux sur la logique, le mathématicien britannique Georges Boole avait travaillé sur des variables ne pouvant prendre que deux valeurs 0  ou 1.
			\item<4-> On appelle, ces variables des \textcolor{red}{booléens}. On définit trois opérations de base que nous allons détailler sur les booléens : le \textcolor{red}{non}, le \textcolor{red}{et} et le \textcolor{red}{ou}.
		\end{itemize}
	\end{block}
\end{frame}




% Opérateur non
\begin{frame}
	\mframe{\Arch}
	\begin{alertblock}{Opérateur \textbf{non}}
		\begin{itemize}
			\item<1-> Inverse la valeur de l'entrée
			\item<2-> Symbole électronique
			      \begin{center}
				      \begin{tabularx}{0.8\textwidth}{Y|Y}
					      \begin{circuitikz} \draw
						      node[american not port](t1) {}
						      ;\end{circuitikz} &
					      \begin{circuitikz} \draw
						      node[european not port](t1) {}
						      ;\end{circuitikz}            \\
					      Américain                 & Européen \\
				      \end{tabularx}
			      \end{center}
			\item<3-> Table de vérité
			      \begin{center}
				      \begin{tabular}{|>{\color{blue}}c|>{\color{red}}c|}
					      \hline
					      Entrée & Sortie \\
					      \hline
					      0      & 1      \\
					      \hline
					      1      & 0      \\
					      \hline
				      \end{tabular}
			      \end{center}
			\item<4> Correspond au \texttt{not} de Python
		\end{itemize}
	\end{alertblock}
\end{frame}

% Opérateur et
\begin{frame}
	\mframe{\Arch}
	\begin{alertblock}{Opérateur \textbf{et}}
		\begin{itemize}
			\item<1-> Vaut 1 lorsque les \textit{deux} entrées valent un
			\item<2-> Symbole électronique
			      \begin{center}
				      \begin{tabularx}{0.8\textwidth}{Y|Y}
					      \begin{circuitikz} \draw
						      node[american and port](t1) {}
						      ;\end{circuitikz} &
					      \begin{circuitikz} \draw
						      node[european and port](t1) {}
						      ;\end{circuitikz}            \\
					      Américain                 & Européen \\
				      \end{tabularx}
			      \end{center}
			\item<3-> Table de vérité
			      \begin{center}
				      \begin{tabular}{|>{\color{blue}}c|>{\color{blue}}c|>{\color{red}}c|}
					      \hline
					      Entrée 1 & Entrée 2 & Sortie \\
					      \hline
					      0        & 0        & 0      \\
					      \hline
					      1        & 0        & 0      \\
					      \hline
					      0        & 1        & 0      \\
					      \hline
					      1        & 1        & 1      \\
					      \hline
				      \end{tabular}
			      \end{center}
			\item<4-> Correspond  au \texttt{and} de Python
		\end{itemize}
	\end{alertblock}
\end{frame}


% Opérateur or
\begin{frame}
	\mframe{\Arch}
	\begin{alertblock}{Opérateur \textbf{or}}
		\begin{itemize}
			\item<1-> Vaut 1 lorsque l'une des deux entrées vaut 1
			\item<2-> Symbole électronique
			      \begin{center}
				      \begin{tabularx}{0.8\textwidth}{Y|Y}
					      \begin{circuitikz} \draw
						      node[american or port](t1) {}
						      ;  \end{circuitikz} &
					      \begin{circuitikz} \draw
						      node[european or port](t1) {}
						      ; \end{circuitikz}            \\
					      Américain                  & Européen \\
				      \end{tabularx}
			      \end{center}
			\item<3-> Table de vérité
			      \begin{center}
				      \begin{tabular}{|>{\color{blue}}c|>{\color{blue}}c|>{\color{red}}c|}
					      \hline
					      Entrée 1 & Entrée 2 & Sortie \\
					      \hline
					      0        & 0        & 0      \\
					      \hline
					      1        & 0        & 1      \\
					      \hline
					      0        & 1        & 1      \\
					      \hline
					      1        & 1        & 1      \\
					      \hline
				      \end{tabular}
			      \end{center}
			\item<4-> Correspond au  \texttt{or} de Python
		\end{itemize}
	\end{alertblock}
\end{frame}

% Autres portes : NAND et XOR
\begin{frame}
	\mframe{\Arch}
	\begin{block}{Autres portes logiques}
		Deux autres portes logiques sont fondamentales et bien que pouvant être construire à partir de OR, AND et NOT ont leur propre symbole :
		\begin{itemize}
			\item<2-> La porte XOR qui vaut 1 lorsque l'une des entrées vaut un mais pas les deux à la fois. C'est le ou exclusif.
			\item<3-> La porte NAND qui vaut 0 seulement lorsque les deux entrées valent 1. C'est la porte "NON ET"
			\item<4-> La porte NOR qui vaut 1 seulement lorsque les deux entrées valent 0. C'est la porte "NON OU"
		\end{itemize}
	\end{block}
\end{frame}


%Python et les booléens
\begin{frame}
	\mframe{\Arch}
	\begin{block}{Python et les booléens}
		\begin{itemize}
			\item<1-> Python possède le type de variable booléen, les deux valeurs possibles sont : \texttt{True} et \texttt{False}.
			\item<2-> L'opération \textbf{non} s'obtient à l'aide de \texttt{not}
			\item<3-> L'opération \textbf{et} s'obtient à l'aide de \texttt{and}
			\item<4-> L'opération \textbf{ou} s'obtient à l'aide de \texttt{or}
			\item<5-> Les booléens de python peuvent donc être notamment des résultats de test de condition.
		\end{itemize}
	\end{block}
	\onslide<6->{
		\begin{exampleblock}{Exemple}
			\texttt{\# Définit une variable booléen ok  qui vaut vrai} \\
			\texttt{\# lorsque au moins 2 des 3 variables a,b et c sont égales}\\
			\onslide<7->{\texttt{ok=(a==b) or (a==c) or (b==c)}}
		\end{exampleblock}}
\end{frame}

% Combinaison et réalisation d'opérations
\begin{frame}
	\mframe{\Arch}
	\begin{alertblock}{Circuit logique}
		\begin{itemize}
			\item<1-> En combinant ces portes logiques, on réalise des circuits logiques permettant d'effectuer des opérations (additions, soustractions, comparaison, ...) sur les données stockées dans l'ordinateur.
			\item<2-> Voir TP sur le site de simulation de circuit logique : {\tt https://circuitverse.org/}
		\end{itemize}
	\end{alertblock}
\end{frame}



\end{document}