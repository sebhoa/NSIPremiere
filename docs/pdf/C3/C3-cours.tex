\PassOptionsToPackage{dvipsnames,table}{xcolor}
\documentclass[10pt]{beamer}
\usepackage{Cours}

\begin{document}

\input{\detokenize{/home/fenarius/Travail/Cours/NSIPremiere/docs/commun/MacrosCours.tex}}
\setcounter{numchap}{3}
\newcommand{\Python}{\cnum Initiation à Python avec turtle}

\pythonmode



% Turtle : papier et crayon
\begin{frame}[fragile]
	\mframe{\Python}
	\begin{block}{Création du papier et du crayon}
		\begin{lstlisting}
	import turtle
	papier = turtle.Screen()
	crayon = turtle.Turtle()
		\end{lstlisting}
	\end{block}
	\onslide<2->{\begin{block}{Remarques}
			\begin{itemize}
				\item<3-> les noms {\tt papier} et {\tt crayon} sont des \textcolor{blue}{noms de variables}, choisis par le programmeur.
				\item<4-> On peut créer plusieurs crayons différents ({\tt stylo =}\tmc{turtle.Turtle()}).
				\item<5-> l'instruction {\tt crayon.}\tmc{reset()} permet d'effacer la totalité des tracés de la tortue nommée {\tt crayon}.
				\item<6-> l'instruction {\tt crayon.}\tmc{undo()} permet d'effacer le dernier tracé de la tortue nommée {\tt crayon}.
			\end{itemize}
		\end{block}}
\end{frame}

% Turtle : papier et crayon 1/2
\begin{frame}[fragile]
	\mframe{\Python}
	\begin{alertblock}{\textcolor{yellow}{\danger} Attention}
		Dans la suite, on supposera que la tortue a été nommée {\tt crayon} et l'écran {\tt papier}. Mais, ces noms sont \textcolor{blue}{choisis par le programmeur}.
	\end{alertblock}
	\begin{block}{Propriétés de la tortue}
		\begin{itemize}
			\item<2-> {\tt crayon.}\tmc{pensize}{\tt(size)} fixe l'épaisseur de la tortue à {\tt size}.
			\item<3-> {\tt crayon.}\tmc{color}{\tt(color)} change à {\tt color} la couleur de la tortue.
			\item<4-> {\tt crayon.}\tmc{penup}{\tt ()} et {\tt crayon.}\tmc{penup}{\tt ()} permettent respectivement de relever ou d'abaisser la tortue.
			\item<5-> {\tt crayon.}\tmc{showturtle}{\tt ()} et {\tt crayon.}\tmc{hideturtle}{\tt ()} permettent respectivement de faire apparaître ou non la tortue.
			\item<6-> {\tt crayon.}\tmc{speed}{\tt (s)}  pour modifier la vitesse de tracé.
		\end{itemize}
	\end{block}
\end{frame}

% Turtle : papier et crayon 2/2
\begin{frame}[fragile]
	\mframe{\Python}
	\begin{exampleblock}{Exemples}
		Ecrire les instructions permettant d'obtenir un crayon abaissé, rouge, d'épaisseur~3, caché et se déplaçant à la vitesse maximale.\pause
		\begin{lstlisting}
	crayon.pendown() 
	crayon.pensize(3)
	crayon.color("red")
	crayon.hideturtle()
	crayon.speed(10)
	\end{lstlisting}
	\end{exampleblock}
\end{frame}


% Turtle : Orientation
\begin{frame}[fragile]
	\mframe{\Python}
	\begin{block}{Orientation de la tortue}
		\begin{itemize}
			\item<2-> L'orientation de la tortue est définie par l'angle qu'elle fait avec l'axe horizontale et est initialement fixé à 0.
			\item<4-> Les instructions suivantes permettent de modifier cette orientation \\
			      \bigskip
			      \begin{tabularx}{\linewidth}{p{4.5cm}|X}
				      \psset{unit=0.6px, xlabelsep}
				      \begin{pspicture}
					      \pstGeonode[PointName=none,PointSymbol=none](90,-70){O}(0,-70){A}(180,-70){B}(90,-160){C}(90,20){D}
					      \ncline[linewidth=0.7px,linecolor=red]{->}{O}{B}
					      \ncline[linewidth=0.5px]{->}{O}{A}
					      \ncline[linewidth=0.5px]{->}{O}{C}
					      \ncline[linewidth=0.5px]{->}{O}{D}
					      \rput(190,-70){{\footnotesize \textcolor{blue}{0}}}
					      \rput(-10,-70){\footnotesize \textcolor{blue}{180}}
					      \rput(90,25){\footnotesize \textcolor{blue}{90}}
					      \rput(90,-170){\footnotesize \textcolor{blue}{270}}
					      \pstRotation[PointName=none,PointSymbol=none,RotAngle=60]{O}{B}[E]
					      \pstRotation[PointName=none,PointSymbol=none,RotAngle=160]{O}{B}[F]
					      \pstRotation[PointName=none,PointSymbol=none,RotAngle=20]{O}{B}[G]
					      \onslide<6->{\ncline[linewidth=0.5px,linecolor=RawSienna]{->}{O}{E} \pstMarkAngle[MarkAngleRadius=40,LabelSep=50,MarkAngleType=default,arrows=->]{B}{O}{E}{$a$}}
					      \onslide<8,9>{\ncline[linewidth=0.5px,linecolor=RawSienna,linestyle=dashed]{->}{O}{F} \pstMarkAngle[MarkAngleRadius=30,LabelSep=20,MarkAngleType=default,arrows=->,linecolor=OliveGreen]{E}{O}{F}{\textcolor{OliveGreen}{$l$}}}{}
					      \onslide<10>{\ncline[linewidth=0.5px,linecolor=RawSienna,linestyle=dashed]{->}{O}{G} \pstMarkAngle[MarkAngleRadius=30,LabelSep=20,MarkAngleType=default,arrows=<-,linecolor=OliveGreen]{G}{O}{E}{\textcolor{OliveGreen}{$r$}}}{}

				      \end{pspicture} &
				      \begin{itemize}
					      \item<5-> {\tt crayon.}\tmc{setheading}{\tt (a)} pour fixer l'orientation de la tortue à l'angle {\tt a}.
					      \item<7-> {\tt crayon.}\tmc{left}{\tt (l)} pour faire tourner la tortue de {\tt l} degrés à gauche \textcolor{blue}{à partir de son orientation actuelle}.
					      \item<9-> {\tt crayon.}\tmc{right}{\tt (r)} pour faire tourner la tortue de {\tt r} degrés à droite \textcolor{blue}{à partir de son orientation actuelle}.
				      \end{itemize}
			      \end{tabularx}
			      \bigskip
		\end{itemize}
	\end{block}
\end{frame}

% Turtle : Déplacement
\begin{frame}[fragile]
	\mframe{\Python}
	\begin{block}{Déplacement de la tortue}
		\begin{itemize}
			\item<2-> La position de la tortue est définie par ses coordonnées dans un repère (comme en mathématiques) et est initialement l'origine du repère.
			\item<4-> Les instructions suivantes permettent de modifier cette position \\
			      \begin{tabularx}{\linewidth}{p{4.5cm}|X}
				      \psset{xunit=0.5cm,yunit=0.5cm,xlabelsep=0,ylabelsep=0.1,MarkHashLength=2pt}
				      \begin{pspicture}(0,4)(-4,4)
					      \psgrid[xunit=0.5cm,yunit=0.5cm,subgriddiv=0,gridlabels=0,gridcolor=lightgray](0,0)(-4,-4)(4,4)
					      \pstGeonode[PointName=none,PointSymbol=none](-4,0){X1}(4,0){X2}(0,-4){Y1}(0,4){Y2}
					      \ncline[linewidth=0.5px]{->}{X1}{X2}
					      \ncline[linewidth=0.5px]{->}{Y1}{Y2}
					      \onslide<6->{\pstGeonode[PointName=none,PointSymbol=|](-2,0){x} \rput(-2,-0.4){{\scriptsize \textcolor{blue}{x}}}}
					      \onslide<6->{\pstGeonode[PointName=none,PointSymbol=+](0,3){y} \rput(0.2,3){{\scriptsize \textcolor{blue}{y}}}}
					      \onslide<7->{\psline[linecolor=RawSienna,linestyle=dashed]{<-}(-2,3)(0,3) \psline[linecolor=RawSienna,linestyle=dashed]{<-}(-2,3)(-2,0) }
					      \onslide<8->{\pstGeonode[PosAngle=135,PointSymbol=*,PointName=none](-2,3){M}}
					      \onslide<10->{\pstGeonode[PosAngle=135,PointSymbol=none,PointName=none](-1.8,2.8){N} \ncline[linewidth=1.5px,linecolor=gray]{->}{M}{N}}
					      \onslide<11->{\pstGeonode[PosAngle=135,PointSymbol=none,PointName=none](2,-1){P} \ncline[linestyle=dashed,linecolor=OliveGreen]{->}{M}{P} \naput{$l$}}
				      \end{pspicture} &
				      \begin{itemize}
					      \item<5-> {\tt crayon.}\tmc{goto}{\tt (x,y)} pour déplacer la tortue au point de coordonnées {\tt (x,y)}.
					      \item<9-> {\tt crayon.}\tmc{forward}{\tt (l)} pour faire avancer la tortue d'une distance {\tt l} \textcolor{blue}{dans sa direction actuelle}.
					      \item<12-> {\tt crayon.}\tmc{backward}{\tt (l)} pour faire reculer la tortue d'une distance {\tt l}\textcolor{blue}{dans la direction opposée à sa direction actuelle}.
				      \end{itemize}
			      \end{tabularx}
			      \bigskip
		\end{itemize}
	\end{block}
\end{frame}


% Appel à une fonction
\begin{frame}[fragile]
	\mframe{\Python}
	\begin{alertblock}{Fonctions}
		\begin{itemize}
			\item<2->{Les fonctions sont des blocs d'instructions destinés à accomplir une tâche lors de leur \textcolor{blue}{appel} (par exemple avec \tmc{turtle}, tracé un carré).}
			\item<3->{Leurs résultats peut dépendre de valeurs appelées \textcolor{blue}{paramètres} de la fonction (par exemple, le côté du carré).}
			\item<4->{Lorsqu'une fonction est destinée à produire un résultat (par exemple celui d'un calcul), on renvoie ce résultat à l'aide de l'instruction \textcolor{blue}{return}}
			\item<5->{Pour définir une fonction en Python, on utilise la syntaxe suivante :
		\begin{lstlisting}
		def <nom_fonction>(<arguments>):
			<instruction>
			return <resultat>
		\end{lstlisting}}
		\end{itemize}
	\end{alertblock}
\end{frame}




\end{document}
