\documentclass[11pt,a4paper]{article}

\usepackage{Act}

\begin{document}
\input{\detokenize{/home/fenarius/Travail/Cours/NSIPremiere/docs/commun/Macros.tex}}
\ModeActivite
\Activites{\algo\; -- C6 : Notions d'algorithmique}{\Pre}
\pythonmode

%Nom de la première activité
\begin{Exercise}[title={Correction d'un algorithme},number=2]

\Question Fonction de recherche du minimum\\
On considère la fonction suivante :
\begin{lstlisting}
def minimum(liste):
    mini = liste[0]
    for k in range(1,len(liste)):
        if liste[k]<mini:
            mini=liste[k]
    return mini
\end{lstlisting}
\subQuestion Que fait cette fonction ?  
\subQuestion Ecrire une chaîne de documentation pour cette fonction.
\subQuestion Ajouter les préconditions suivantes sous forme d'instructions \texttt{assert} : la liste passée en argument est non vide et ne contient que des valeurs de type entier ou flottant.
\subQuestion Proposer un jeu de tests sous forme d'instruction \texttt{assert} pour cette fonction.
\Question Problème de la correction \\
\danger \; Le terme \og correction \fg \; ne désigne pas ici l'action de corriger, on dit qu'un algorithme est \textbf{correct} lorsqu'il fournit bien la réponse attendue à un problème dans tous les cas.

\subQuestion On note $e_0, e_1, e_2, ...$ les éléments de la liste passée en paramètre. Que contient la variable \texttt{mini} avant d'entrer dans la boucle \texttt{for} ? Le donner sous la forme du minimum d'une liste d'éléments.
\subQuestion Donner sous la forme du minimum d'une liste d'éléments, le contenu de la variable \texttt{mini} après un passage dans la boucle.
\subQuestion On cherche à prouver que notre fonction est \textit{correcte} c'est à dire qu'elle renvoie le minimum de \textbf{n'importe} quelle liste passée en paramètre. Les tests effectués suffisent-ils ?
\subQuestion Même question après deux passages, après trois passages.
\subQuestion Proposer une propriété portant sur la variable \texttt{mini} et qui reste vraie avant d'entrer dans la boucle et à chaque itération dans la boucle.
\subQuestion Montrer que cette propriété est vraie. \\
\aide \; On prouvera successivement que :
 \begin{itemize}
 \item La propriété est vraie avant d'entrer dans la boucle. 
 \item Si la propriété est vraie  lors d'une itération alors elle reste vraie à l'itération suivante.
 \end{itemize}
\Question Conclure sur la correction de la fonction.
\end{Exercise}
\separateur

\begin{Exercise}[title={Terminaison d'un algorithme}]

\Question Multiplier deux entiers en faisant des additions \\
On considère la fonction suivante :
\begin{lstlisting}
def mult(n,p):
    produit = 0
    k = p
    while k>0:
        k=k-1
        produit = produit+n
    return produit
\end{lstlisting}

\subQuestion Que fait cette fonction ?  
\subQuestion Ecrire une chaîne de documentation pour cette fonction.
\subQuestion Ajouter les préconditions suivantes sous forme d'instructions \texttt{assert} : les arguments sont des entiers positifs.
\subQuestion Proposer un jeu de tests sous forme d'instruction \texttt{assert} pour cette fonction.

\Question Terminaison de cet algorithme

\subQuestion Quels sont les valeurs successives prises par la variable \texttt{k}  lors de chaque passage dans la boucle ?
\subQuestion Conclure. \\
\aide \; On pourra utiliser la propriété mathématique suivante : \og il n'existe pas de suite d'entiers naturels strictement décroissante \fg. 
\end{Exercise}
\end{document}