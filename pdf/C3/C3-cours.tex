\PassOptionsToPackage{dvipsnames,table}{xcolor}
\documentclass[10pt]{beamer}
\usepackage{Cours}

\begin{document}

\input{\detokenize{/home/fenarius/Travail/Cours/NSIPremiere/docs/commun/MacrosCours.tex}}
\setcounter{numchap}{3}
\newcommand{\Python}{\cnum Initiation à Python avec turtle}

\pythonmode


% Remarques sur les langages de programmation
\begin{frame}
	\mframe{\Python}
	\begin{block}{Remarques}
		\begin{itemize}
			\item<1-> Il existe \textit{plusieurs milliers} de langages informatique.
			\item<2-> Parmi les plus importants historiquement et les plus utilisés, on peut citer :
			      \begin{itemize}
				      \item<3-> Cobol, Fortran, Lisp,
				      \item<4-> le \textcolor{blue}{langage C} et ses dérivés, Pascal, Ada
				      \item<5-> Perl, Javascript, PHP, \textcolor{red}{Python}, Java.
			      \end{itemize}
			\item<6-> Ne pas confondre langages de programmations et algorithme. Un algorithme est une méthode permettant de résoudre un problème. Un langage informatique permet de formuler un algorithme pour l'exécuter sur un ordinateur.
		\end{itemize}
	\end{block}
\end{frame}


% Python
\begin{frame}
	\mframe{\Python}
	\begin{alertblock}{A propos de Python}
		\begin{itemize}
			\item<1-> Langage crée par un informaticien néerlandais : \textcolor{blue}{Guido Van Rossum}. La première version date de 1991, nous en sommes actuellement à la version 3.
			\item<2-> C'est un langage sous licence libre, gratuit, multiplateforme.
			\item<3-> C'est un langage extensible grâce à l'ajout de nombreuses bibliothèques.
			\item<4-> C'est un langage interprété.
			\item<5-> Choisi pour l'enseignement de la spécialité {\sc nsi} au lycée.
		\end{itemize}
	\end{alertblock}
\end{frame}


% Variables
\begin{frame}[fragile]
	\mframe{\Python}
	\begin{block}{Programmation en python : généralités}
		\begin{itemize}
			\item<1-> Python renvoie un message d'erreur lorsqu'il n'arrive pas à interpréter les instructions de votre programme. Prendre l'habitude de \textcolor{blue}{lire attentivement} ces messages, qui sont de premiers indices pour déterminer la source de l'erreur
			\item<2-> En Python, les \textcolor{blue}{commentaires} s'écrivent en faisant commencer la ligne par le caractère \textcolor{red}{\tt \#}.
			\item<3-> Le respect de la syntaxe du langage est \textcolor{blue}{fondamental} et demande de la \textcolor{blue}{rigueur}.
			\item<4-> Attention aussi à bien surveiller les correspondances entre les parenthèses mais aussi les guillemets ou les apostrophes qui sont souvent source d'erreurs.
		\end{itemize}
	\end{block}
\end{frame}

% Turtle : papier et crayon
\begin{frame}[fragile]
	\mframe{\Python}
	\begin{block}{Création du papier et du crayon}
		\begin{lstlisting}
	import turtle
	papier = turtle.Screen()
	crayon = turtle.Turtle()
		\end{lstlisting}
	\end{block}
	\onslide<2->{\begin{block}{Remarques}
			\begin{itemize}
				\item<3-> les noms {\tt papier} et {\tt crayon} sont des \textcolor{blue}{noms de variables}, choisis par le programmeur.
				\item<4-> On peut créer plusieurs crayons différents ({\tt stylo =}\tmc{turtle.Turtle()}).
				\item<5-> l'instruction {\tt crayon.}\tmc{reset()} permet d'effacer la totalité des tracés de la tortue nommée {\tt crayon}.
				\item<6-> l'instruction {\tt crayon.}\tmc{undo()} permet d'effacer le dernier tracé de la tortue nommée {\tt crayon}.
			\end{itemize}
		\end{block}}
\end{frame}

% Turtle : papier et crayon 1/2
\begin{frame}[fragile]
	\mframe{\Python}
	\begin{alertblock}{\textcolor{yellow}{\danger} Attention}
		Dans la suite, on supposera que la tortue a été nommée {\tt crayon} et l'écran {\tt papier}. Mais, ces noms sont \textcolor{blue}{choisis par le programmeur}.
	\end{alertblock}
	\begin{block}{Propriétés de la tortue}
		\begin{itemize}
			\item<2-> {\tt crayon.}\tmc{pensize}{\tt(size)} fixe l'épaisseur de la tortue à {\tt size}.
			\item<3-> {\tt crayon.}\tmc{color}{\tt(color)} change à {\tt color} la couleur de la tortue.
			\item<4-> {\tt crayon.}\tmc{penup}{\tt ()} et {\tt crayon.}\tmc{penup}{\tt ()} permettent respectivement de relever ou d'abaisser la tortue.
			\item<5-> {\tt crayon.}\tmc{showturtle}{\tt ()} et {\tt crayon.}\tmc{hideturtle}{\tt ()} permettent respectivement de faire apparaître ou non la tortue.
			\item<6-> {\tt crayon.}\tmc{speed}{\tt (s)}  pour modifier la vitesse de tracé.
		\end{itemize}
	\end{block}
\end{frame}

% Turtle : papier et crayon 2/2
\begin{frame}[fragile]
	\mframe{\Python}
	\begin{exampleblock}{Exemples}
		Ecrire les instructions permettant d'obtenir un crayon abaissé, rouge, d'épaisseur~3, caché et se déplaçant à la vitesse maximale.\pause
		\begin{lstlisting}
	crayon.pendown() 
	crayon.pensize(3)
	crayon.color("red")
	crayon.hideturtle()
	crayon.speed(10)
	\end{lstlisting}
	\end{exampleblock}
\end{frame}


% Turtle : Orientation
\begin{frame}[fragile]
	\mframe{\Python}
	\begin{block}{Orientation de la tortue}
		\begin{itemize}
			\item<2-> L'orientation de la tortue est définie par l'angle qu'elle fait avec l'axe horizontale et est initialement fixé à 0.
			\item<4-> Les instructions suivantes permettent de modifier cette orientation \\
			      \bigskip
			      \begin{tabularx}{\linewidth}{p{4.5cm}|X}
				      \psset{unit=0.6px, xlabelsep}
				      \begin{pspicture}
					      \pstGeonode[PointName=none,PointSymbol=none](90,-70){O}(0,-70){A}(180,-70){B}(90,-160){C}(90,20){D}
					      \ncline[linewidth=0.7px,linecolor=red]{->}{O}{B}
					      \ncline[linewidth=0.5px]{->}{O}{A}
					      \ncline[linewidth=0.5px]{->}{O}{C}
					      \ncline[linewidth=0.5px]{->}{O}{D}
					      \rput(190,-70){{\footnotesize \textcolor{blue}{0}}}
					      \rput(-10,-70){\footnotesize \textcolor{blue}{180}}
					      \rput(90,25){\footnotesize \textcolor{blue}{90}}
					      \rput(90,-170){\footnotesize \textcolor{blue}{270}}
					      \pstRotation[PointName=none,PointSymbol=none,RotAngle=60]{O}{B}[E]
					      \pstRotation[PointName=none,PointSymbol=none,RotAngle=160]{O}{B}[F]
					      \pstRotation[PointName=none,PointSymbol=none,RotAngle=20]{O}{B}[G]
					      \onslide<6->{\ncline[linewidth=0.5px,linecolor=RawSienna]{->}{O}{E} \pstMarkAngle[MarkAngleRadius=40,LabelSep=50,MarkAngleType=default,arrows=->]{B}{O}{E}{$a$}}
					      \onslide<8,9>{\ncline[linewidth=0.5px,linecolor=RawSienna,linestyle=dashed]{->}{O}{F} \pstMarkAngle[MarkAngleRadius=30,LabelSep=20,MarkAngleType=default,arrows=->,linecolor=OliveGreen]{E}{O}{F}{\textcolor{OliveGreen}{$l$}}}{}
					      \onslide<10>{\ncline[linewidth=0.5px,linecolor=RawSienna,linestyle=dashed]{->}{O}{G} \pstMarkAngle[MarkAngleRadius=30,LabelSep=20,MarkAngleType=default,arrows=<-,linecolor=OliveGreen]{G}{O}{E}{\textcolor{OliveGreen}{$r$}}}{}

				      \end{pspicture} &
				      \begin{itemize}
					      \item<5-> {\tt crayon.}\tmc{setheading}{\tt (a)} pour fixer l'orientation de la tortue à l'angle {\tt a}.
					      \item<7-> {\tt crayon.}\tmc{left}{\tt (l)} pour faire tourner la tortue de {\tt l} degrés à gauche \textcolor{blue}{à partir de son orientation actuelle}.
					      \item<9-> {\tt crayon.}\tmc{right}{\tt (r)} pour faire tourner la tortue de {\tt r} degrés à droite \textcolor{blue}{à partir de son orientation actuelle}.
				      \end{itemize}
			      \end{tabularx}
			      \bigskip
		\end{itemize}
	\end{block}
\end{frame}

% Turtle : Déplacement
\begin{frame}[fragile]
	\mframe{\Python}
	\begin{block}{Déplacement de la tortue}
		\begin{itemize}
			\item<2-> La position de la tortue est définie par ses coordonnées dans un repère (comme en mathématiques) et est initialement l'origine du repère.
			\item<4-> Les instructions suivantes permettent de modifier cette position \\
			      \begin{tabularx}{\linewidth}{p{4.5cm}|X}
				      \psset{xunit=0.5cm,yunit=0.5cm,xlabelsep=0,ylabelsep=0.1,MarkHashLength=2pt}
				      \begin{pspicture}(0,4)(-4,4)
					      \psgrid[xunit=0.5cm,yunit=0.5cm,subgriddiv=0,gridlabels=0,gridcolor=lightgray](0,0)(-4,-4)(4,4)
					      \pstGeonode[PointName=none,PointSymbol=none](-4,0){X1}(4,0){X2}(0,-4){Y1}(0,4){Y2}
					      \ncline[linewidth=0.5px]{->}{X1}{X2}
					      \ncline[linewidth=0.5px]{->}{Y1}{Y2}
					      \onslide<6->{\pstGeonode[PointName=none,PointSymbol=|](-2,0){x} \rput(-2,-0.4){{\scriptsize \textcolor{blue}{x}}}}
					      \onslide<6->{\pstGeonode[PointName=none,PointSymbol=+](0,3){y} \rput(0.2,3){{\scriptsize \textcolor{blue}{y}}}}
					      \onslide<7->{\psline[linecolor=RawSienna,linestyle=dashed]{<-}(-2,3)(0,3) \psline[linecolor=RawSienna,linestyle=dashed]{<-}(-2,3)(-2,0) }
					      \onslide<8->{\pstGeonode[PosAngle=135,PointSymbol=*,PointName=none](-2,3){M}}
					      \onslide<10->{\pstGeonode[PosAngle=135,PointSymbol=none,PointName=none](-1.8,2.8){N} \ncline[linewidth=1.5px,linecolor=gray]{->}{M}{N}}
					      \onslide<11->{\pstGeonode[PosAngle=135,PointSymbol=none,PointName=none](2,-1){P} \ncline[linestyle=dashed,linecolor=OliveGreen]{->}{M}{P} \naput{$l$}}
				      \end{pspicture} &
				      \begin{itemize}
					      \item<5-> {\tt crayon.}\tmc{goto}{\tt (x,y)} pour déplacer la tortue au point de coordonnées {\tt (x,y)}.
					      \item<9-> {\tt crayon.}\tmc{forward}{\tt (l)} pour faire avancer la tortue d'une distance {\tt l} \textcolor{blue}{dans sa direction actuelle}.
					      \item<12-> {\tt crayon.}\tmc{backward}{\tt (l)} pour faire reculer la tortue d'une distance {\tt l}\textcolor{blue}{dans la direction opposée à sa direction actuelle}.
				      \end{itemize}
			      \end{tabularx}
			      \bigskip
		\end{itemize}
	\end{block}
\end{frame}


% Appel à une fonction
\begin{frame}[fragile]
	\mframe{\Python}
	\begin{alertblock}{Fonctions}
		\begin{itemize}
			\item<2->{Les fonctions sont des blocs d'instructions destinés à accomplir une tâche lors de leur \textcolor{blue}{appel} (par exemple avec \tmc{turtle}, tracé un carré).}
			\item<3->{Leurs résultats peut dépendre de valeurs appelées \textcolor{blue}{paramètres} de la fonction (par exemple, le côté du carré).}
			\item<4->{Pour définir une fonction en Python, on utilise la syntaxe suivante :
			      \begin{lstlisting}
	def <nom_fonction>(<arguments>):
		<instruction>
		\end{lstlisting}}
		\end{itemize}
	\end{alertblock}
\end{frame}

% Exemple de fonction
\begin{frame}[fragile]
	\mframe{\Python}
	\begin{exampleblock}{Un exemple de fonction}
		Tracer un trait à partir du point de coordonnées {\tt (x,y)} en donnant la longueur $l$ et la direction $a$.
		\begin{tabularx}{\linewidth}{p{4.5cm}|X}
			\psset{xunit=0.5cm,yunit=0.5cm,xlabelsep=0,ylabelsep=0.1,MarkHashLength=2pt}
			\begin{pspicture}(0,2)(-4,2)
				\pstGeonode[PointName=none,PointSymbol=none](-2,-3){M}(2,-3){MX}(1,2){N}
				\rput(-3,-3){(x,y)} \rput(0,-2.5){Angle $a$}
				\ncline[linewidth=1px,linecolor=blue]{->}{M}{N} \naput{longueur $l$}
				\ncline[linewidth=0.25px,linestyle=dashed]{->}{M}{MX}
				\pstMarkAngle[LabelSep=50,arrows=->]{MX}{M}{N}{Angle $a$}
			\end{pspicture} &
			\begin{itemize}
				\item<2-> Relever le crayon
				\item<3-> Aller en {\tt (x,y)}
				\item<4-> S'orienter dans la direction d'angle $a$
				\item<5-> Abaisser le crayon
				\item<6-> Avancer de la longueur $l$
			\end{itemize}
		\end{tabularx}
	\end{exampleblock}
\end{frame}

% Exemple de fonction
\begin{frame}[fragile]
	\mframe{\Python}
	\begin{exampleblock}{Un exemple de fonction}
		Tracer un trait à partir du point de coordonnées {\tt (x,y)} en donnant la longueur $l$ et la direction $a$.
		\begin{tabularx}{\linewidth}{p{4.5cm}|X}
			\psset{xunit=0.5cm,yunit=0.5cm,xlabelsep=0,ylabelsep=0.1,MarkHashLength=2pt}
			\begin{pspicture}(0,2)(-4,2)
				\pstGeonode[PointName=none,PointSymbol=none](-2,-3){M}(2,-3){MX}(1,2){N}
				\rput(-3,-3){(x,y)} \rput(0,-2.5){Angle $a$}
				\ncline[linewidth=1px,linecolor=blue]{->}{M}{N} \naput{longueur $l$}
				\ncline[linewidth=0.25px,linestyle=dashed]{->}{M}{MX}
				\pstMarkAngle[LabelSep=50,arrows=->]{MX}{M}{N}{Angle $a$}
			\end{pspicture} &
			\begin{itemize}
				\item Relever le crayon
				\item Aller en {\tt (x,y)}
				\item S'orienter dans la direction d'angle $a$
				\item Abaisser le crayon
				\item Avancer de la longueur $l$
			\end{itemize}
		\end{tabularx}
		\begin{lstlisting}
		def trait(x,y,l,a):
			crayon.penup()
			crayon.goto(x,y)
			crayon.setheading(a)
			crayon.pendown()
			crayon.forward(l)
		\end{lstlisting}
	\end{exampleblock}
\end{frame}


% boucle for
\begin{frame}[fragile]
	\mframe{\Python}
	\begin{alertblock}{Boucles {\tt for}}
		\begin{itemize}
			\item<2-> Les instructions :
			      \begin{lstlisting}
	for <variable> in range(<entier>):
		 <instructions>
	\end{lstlisting}
			      permet de créer une variable parcourant les entiers de 0 à {\tt <entier>} (exclu).
			\item<3-> Les {\tt <instructions>} indentées qui suivent seront executées pour chaque valeur prise par la variable.
			\item<4-> La boucle {\tt for} permet donc de répéter un nombre prédéfini de fois des instructions, on dit que c'est une boucle bornée.
		\end{itemize}
	\end{alertblock}
\end{frame}

% boucle for
\begin{frame}[fragile]
	\mframe{\Python}
	\begin{exampleblock}{Une boucle avec la fonction trait}
		En faisant varier la direction et la longueur dans la fonction {\tt trait} définie ci-dessus à l'aide d'une boucle on obtient le dessin suivant :
		\begin{center}
			\includegraphics[scale=0.25]{ex-cours.eps}
		\end{center}
	\begin{lstlisting}
	direction = 0
	longueur = 10
	for t in range(51):
		trait(0,0,longueur,direction)
		direction+=3.6
		longueur+=5
	papier.exitonclick()
	\end{lstlisting}
	\end{exampleblock}
\end{frame}

\end{document}
