\documentclass[11pt,a4paper]{article}

\usepackage{Act}

\begin{document}
\input{\detokenize{/home/fenarius/Travail/Cours/MkDocs/docs/commun/Macros.tex}}
\ModeExercice
\Exos{Systèmes d'exploitation}{\Pre}
\begin{Exercise}[title={\sc qcm},origin={\bac}]
	\Question
	\QNSI{Une et une seule de ces affirmations est fausse. Laquelle ?}
	{Un système d'exploitation libre est la plupart du temps gratuit}
	{Je peux contribuer à un système d'exploitation libre}
	{Il est interdit d'étudier un système d'exploitation propriétaire}
	{Un système d'exploitation propriétaire est plus sécurisé}
	\Question
	\QNSI{Quelle est l'effet de la commande \texttt{cd ..} ?}
	{Changer le répertoire courant vers le répertoire supérieur}
	{Ejecter le {\sc cd}}
	{Copier le contenu du répertoire courant dans un répertoire caché}
	{Supprimer le répertoire courant}
	\Question\QNSI{Dans un terminal sous Linux, quelle commande faut-il écrire pour donner à tout le monde le droit d'écriture sur un fichier \texttt{information.py} ?}
	{\texttt{chmod o+w information.py}}
	{\texttt{chmod a+w information.py}}
	{\texttt{chmod o+x information.py}}
	{\texttt{chmod a+x information.py}}
	\Question\QNSI{Quel est le rôle de la commande \texttt{ls} ?}
	{basculer en mode administrateur}
	{lister le contenu du répertoire courant}
	{effacer le contenu du répertoire courant}
	{donner un accès complet à un fichier}
	\Question\QNSI{Quel est l'effet de la commande \texttt{chmod u+w fichier.txt} ?}
	{de permettre au propriétaire du fichier de modifier le contenu de ce fichier}
	{d'interdire au propriétaire de modifier le contenu de ce fichier}
	{d'interdire à tous les autres utilisateurs de lire le fichier}
	{d'effacer le fichier}
	\Question \QNSI{Dans la console Linux, quelle commande faut-il exécuter pour obtenir la documentation sur la commande pwd ?}
	{\texttt{man pwd}}
	{\texttt{cd pwd}}
	{\texttt{mkdir pwd}}
	{\texttt{ls pwd}}
	\Question \QNSI{Lequel de ces systèmes d'exploitation est sous licence propriétaire ?}
	{Android}
	{Linux}
	{Windows}
	{Ubuntu}
	\Question \QNSI{Une application doit écrire sur le disque dur de l'ordinateur, laquelle des affirmations suivantes est vraie ?}
	{Elle peut le faire quand bon lui semble.}
	{Elle doit en faire la demande au système d'exploitation.}
	{Elle doit en faire la demande aux autres applications qui fonctionnent sur l'ordinateur.}
	{Elle doit en faire la demande à l'utilisateur de l'ordinateur}
\end{Exercise}
\pagebreak

\Exo{Une nouvelle commande}{}
\QListe
\item Essayer la commande \texttt{cal}.
\item Lire la documentation sur cette commande.
\item Quel était le jour de la semaine le 26 juin 1815 ?
\FinListe
\vspace{0.1cm}


\Exo{Premiers pas en Python en ligne de commande}{}\\
Le langage Python peut être invoqué à partir de la ligne de commande, taper simplement 	\texttt{python} dans webminal.
L'invite de commande se transforme en \texttt{>\,>\,>}, on dit que Python est en mode console. Vous pouvez quitter Python en tapant \texttt{exit()}.
\QListe
\item Utiliser Python comme calculatrice \\
En mode console, Python vous fournira directement les résultats de calculs, tester par exemple :
\SQListe
\item \texttt{15+5*5}, dans quel ordre les opérations sont-elles effectuées ?
\item \texttt{2**10}, de quelle opération s'agit-il ? (tester d'autres valeurs si necessaires)
\item \texttt{20\%3}  et  \texttt{20//3}, de quelle opération s'agit-il ? (tester d'autres valeurs si necessaires)
\FinListe
\item Obtenir de l'aide en python
\SQListe
\item Tester les commandes \texttt{chr(33)},  \texttt{chr(72)}, \texttt{chr(125)}
\item Pour connaître l'utilité de cette commande demander \texttt{help(chr)}
\FinListe
\FinListe
\vspace{0.1cm}


\Exo{Ecrire dans un fichier, éditer un fichier}{}\\
Nous avons vu que la command \texttt{touch} permet de créer un fichier vide. Il serait bien sur intéressant d'écrire dans ce fichier, c'est ce que nous allons découvrir dans cet exercice.
Commencer par créer un fichier vide nommé \texttt{monfichier.txt}.
\QListe
\item Avec la commande \texttt{echo}
\SQListe
\item Tester la commande, par exemple en tapant \texttt{echo "Hello world"}
\item Par défaut la commande \texttt{echo} écrit dans le terminal, nous pouvons lui préciser d'écrire dans un fichier à l'aide du caractère \texttt{>}. Par exemple avec \texttt{echo "Hello world" > monfichier.txt}
\item Afficher le contenu du fichier (commande \texttt{cat}) pour constater qu'il a été modifié.
\FinListe
\item Avec un éditeur de texte : nano. Taper la commande \texttt{nano monfichier.txt}.\\
nano est un éditeur de texte fonctionnant dans le terminal. Les commandes principales s'affichent en bas de page (le caractère \texttt{\^} désigne la touche \framebox{{\sc ctrl}}).
\FinListe
\vspace{0.1cm}


\Exo{Faire un peu de rangement}{\bombe}\\
Dans le dossier \texttt{/common\_pool}, se trouve un dossier nommé \texttt{a\_ranger}, comme son nom l'indique ce dossier demande à être rangé car il contient de nombreux fichiers de diverses sortes (texte, page html ou programme python) tous mélangés dans le même dossier.
\QListe
\item Lister le contenu de ce dossier pour découvrir par vous-même son contenu.
\item Copier le dossier \texttt{a\_ranger} ainsi que tous les fichiers qu'il contient dans votre répertoire personnel.\\
\aide \; Consulter l'aide de la commande \texttt{cp} pour découvrir comment copier un dossier et son contenu.
\item Dans cette copie créer trois dossiers nommés \texttt{Texte}, \texttt{Html} et \texttt{Python}. Puis,déplacer les fichiers de chaque type dans le dossier correspondant. \\
\aide \; Déplacer les fichiers un par un serait long et fastidieux, penser à utiliser le caractère  \texttt{*}, en effet dans un nom de fichier il remplace n'importe quelle séquence de caractères. Par exemple \texttt{*.bak} désigne n'importe quel fichier dont le nom se termine par \texttt{.bak}.
\FinListe
\vspace{0.1cm}


\Exo{Enigme}{\bombe \; \recherche}
\QListe
\item Trouver  six lettres  en \textit{utilisant uniquement Webminal en ligne de commande}.
\begin{itemize}
	% U
	\item Lettre n°1 : \og \textit{cachée dans \texttt{/common\_pool/EnigmeNSI}} \fg.
	      % G
	\item Lettre n°2 : \og \textit{son code {\sc ascii} est 71} \fg.
	      % I
	\item Lettre n°3 : \og \textit{deuxième lettre du jour de la semaine du 23 juin 1912} \fg.
	      % N (14 e lettre de l'alphabet).
	\item Lettre n°4 : \og \textit{La commande \texttt{wc} vous permettra de compter le nombre de caractères du fichier lettre4 qui se trouve dans \texttt{/common\_pool/EnigmeNSI}. Diviser le résultat obtenu par 1956 pour avoir la position dans l'alphabet de la quatrième lettre}
	      % R
	\item Lettre n°5 : \og \textit{Lorsque cette lettre est donnée en option à la commande \texttt{cp}, elle permet de copier un dossier et tout ce qu'il contient }\fg
	      % T 20e lettre
	\item Lettre n°6 : \og \textit{La taille en kilo octet du fichier \texttt{last} qui se trouve dans \texttt{/usr/bin} vous donnera le rang dans l'alphabet de la sixième lettre}
\end{itemize}
\item Remettez dans l'ordre les six lettres obtenues pour trouver le nom d'un célèbre informaticien. Faites quelques recherches sur le \textit{Web} pour découvrir ses contributions à l'informatique.
\FinListe
\end{document}