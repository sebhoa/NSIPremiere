\PassOptionsToPackage{dvipsnames,table}{xcolor}
\documentclass[10pt]{beamer}
\usepackage{Cours}

\begin{document}

\input{\detokenize{/home/fenarius/Travail/Cours/NSIPremiere/docs/commun/MacrosCours.tex}}
\setcounter{numchap}{2}
\newcommand{\Encodage}{\cnum Représentation des entiers, encodage des caractères}
\pythonmode

% Ecriture des nombres avec des 0 et des 1
\begin{frame}
	\mframe{\Encodage}
	\begin{alertblock}{Ecriture binaire}
		\begin{itemize}
			\item<1-> On peut écrire les nombres entiers positifs en utilisant seulement deux chiffres : 0 et 1.
			\item<2-> Chaque chiffre est multiplié par une puissance de 2 selon  sa position dans le nombre.
		\end{itemize}
	\end{alertblock}
	\begin{exampleblock}{Exemple}
		\onslide<3->{Par exemple en binaire le nombre \textcolor{red}{$10001011$} correspond à 139 en décimal : \\}
		\onslide<4->{
			\begin{tabular}{p{0.4cm}p{0.4cm}p{0.4cm}p{0.4cm}p{0.4cm}p{0.4cm}p{0.4cm}p{0.4cm}c}
				                   &                    &                    &                    &                    &                    &                    &                    & \\
				\textcolor{red}{1} & \textcolor{red}{0} & \textcolor{red}{0} & \textcolor{red}{0} & \textcolor{red}{1} & \textcolor{red}{0} & \textcolor{red}{1} & \textcolor{red}{1}   \\
			\end{tabular}}
	\end{exampleblock}
\end{frame}


% Ecriture des nombres avec des 0 et des 1
\begin{frame}
	\mframe{\Encodage}
	\begin{alertblock}{Ecriture binaire}
		\begin{itemize}
			\item On peut écrire les nombres entiers positifs en utilisant seulement deux chiffres : 0 et 1.
			\item Chaque chiffre est multiplié par une puissance de 2 selon  sa position dans le nombre.
		\end{itemize}
	\end{alertblock}
	\begin{exampleblock}{Exemple}
		Par exemple en binaire le nombre \textcolor{red}{$10001011$} correspond à 139 en décimal : \\
		\begin{tabular}{|p{0.4cm}|p{0.4cm}|p{0.4cm}|p{0.4cm}|p{0.4cm}|p{0.4cm}|p{0.4cm}|p{0.4cm}|l}
			\textcolor{blue}{$\scriptstyle{2^7}$} & \textcolor{blue}{$\scriptstyle{2^6}$}   & \textcolor{blue}{$\scriptstyle{2^5}$} & \textcolor{blue}{$\scriptstyle{2^4}$} & \textcolor{blue}{$\scriptstyle{2^3}$} & \textcolor{blue}{$\scriptstyle{2^2}$} & \textcolor{blue}{$\scriptstyle{2^1}$} & \textcolor{blue}{$\scriptstyle{2^0}$}                                                                                                                                                                                                                                 \\
			\cline{1-8}
			\textcolor{red}{1}                    & \textcolor{red}{0}                      & \textcolor{red}{0}                    & \textcolor{red}{0}                    & \textcolor{red}{1}                    & \textcolor{red}{0}                    & \textcolor{red}{1}                    & \textcolor{red}{1}                    & = \onslide<2->{\textcolor{red}{1}$\times$ \textcolor{blue}{$2^7$}+\textcolor{red}{1}$\times$ \textcolor{blue}{$2^3$}+\textcolor{red}{1}$\times$ \textcolor{blue}{$2^1$} + \textcolor{red}{1}$\times$ \textcolor{blue}{$2^0$}} \\
			\multicolumn{8}{l}{}                  & \onslide<3->{$= 128 + 8 + 2 + 1 = 139$}                                                                                                                                                                                                                                                                                                                                                                                                                                                                                 \\
		\end{tabular}
	\end{exampleblock}
\end{frame}

% Remarques
\begin{frame}
	\mframe{\Encodage}
	\begin{block}{Remarque sur l'écriture décimale :}
		Nous sommes habitués à écrire les nombres en base 10, et en utilisant 10 chiffres (0,1,2,3,4,5,6,7,8 et 9), mais c'est le \textbf{même} principe qui est utilisé : les chiffres d'un nombre sont multipliés par une puissance de 10 suivant leur emplacement dans le nombre.\\
		\onslide<2->{Par exemple, pour \textcolor{red}{1815} :\\}
		\onslide<3->{\begin{tabular}{p{0.4cm}p{0.4cm}p{0.4cm}p{0.4cm}c}
				                   &                    &                    &                    & \\
				\textcolor{red}{1} & \textcolor{red}{8} & \textcolor{red}{1} & \textcolor{red}{5} & \\
			\end{tabular}
		}
	\end{block}
\end{frame}

\begin{frame}
	\mframe{\Encodage}
	\begin{block}{Remarque sur l'écriture décimale :}
		Nous sommes habitués à écrire les nombres en base 10, et en utilisant 10 chiffres (0,1,2,3,4,5,6,7,8 et 9), mais c'est le \textbf{même} principe qui est utilisé : les chiffres d'un nombre sont multipliés par une puissance de 10 suivant leur emplacement dans le nombre.\\
		Par exemple, pour \textcolor{red}{1815} :\\
		\begin{tabular}{p{0.4cm}|p{0.4cm}|p{0.4cm}|p{0.4cm}c}
			$\scriptstyle{10^3}$ & $\scriptstyle{10^2}$ & $\scriptstyle{10^1}$ & $\scriptstyle{10^0}$ &                                                                             \\
			\cline{1-4}
			\textcolor{red}{1}   & \textcolor{red}{8}   & \textcolor{red}{1}   & \textcolor{red}{5}   & \onslide<2->{$=1 \times 1000 + 8 \times 100 + 1\times 10+ 1 \times 1=1815$} \\
		\end{tabular}
	\end{block}
\end{frame}

% Convention d'écriture 
\begin{frame}
	\mframe{\Encodage}
	\begin{block}{Convention d'écriture}
		\begin{itemize}
			\item<1-> Le nombre $101$ pourrait être écris en base 2 (et donc valoir \onslide<2->{cinq)}
			      \onslide<3->{, ou être écrit en base 10, et donc valoir cent un.}
			\item<4-> Afin d'éviter toute confusion, on convient d'écrire le nombre entre parenthèses et de mettre en indice la base dans lequel il est écrit
			\item<5-> Par exemple $\base{10001}{2}$ est le nombre valant, \onslide<6->{dix-sept.}
			\item<6-> Par contre $\base{10000}{10}$ vaut dix mille.
		\end{itemize}
	\end{block}
\end{frame}

% Vocabulaire
\begin{frame}
	\mframe{\Encodage}
	\begin{block}{Vocabulaire}
		\begin{itemize}
			\item<1-> Un chiffre en base 2 s'appelle un \textcolor{red}{bit}, un bit vaut donc 0 ou 1.
			\item<2-> Le regroupement de 8 bits s'appelle un \textcolor{red}{octet}.
			\item<3-> En utilisant un octet, on peut représenter les entiers de 0 à 255.
		\end{itemize}
	\end{block}
\end{frame}


% Exemples
\begin{frame}
	\mframe{\Encodage}
	\begin{exampleblock}{\textcolor{yellow}{\flash} {Question flash}}
		Compléter le tableau de conversion suivant : \\
		\begin{center}
			\begin{tabularx}{0.6\textwidth}{|X|X|}
				\hline
				Ecriture décimale & Ecriture binaire     \\
				\hline
				$\base{142}{10}$  &                      \\
				\hline
				$\base{207}{10}$  &                      \\
				\hline
				                  & $\base{100101}{2}$   \\
				\hline
				$\base{88}{10}$   &                      \\
				\hline
				$\base{222}{10}$  &                      \\
				\hline
				                  & $\base{11100001}{2}$ \\
				\hline
				                  & $\base{11110}{2}$    \\
				\hline
			\end{tabularx}
		\end{center}\end{exampleblock}
\end{frame}


% Exemples
\begin{frame}
	\mframe{\Encodage}
	\begin{exampleblock}{\textcolor{yellow}{\flash} {Question flash}}
		\begin{itemize}
			\item<1-> Ecrire les entiers positifs de 1 à 16 en base 2 :
			      \renewcommand{\arraystretch}{1.5}
			      \begin{tabularx}{0.92\textwidth}{|X|X|X|X|}
				      \hline
				      $(1)_{10}= (\dots)_2$  & $(2)_{10}= (\dots)_2$  & $(3)_{10}= (\dots)_2$  & $(4)_{10}= (\dots)_2$  \\
				      \hline
				      $(5)_{10}= (\dots)_2$  & $(6)_{10}= (\dots)_2$  & $(7)_{10}= (\dots)_2$  & $(8)_{10}= (\dots)_2$  \\
				      \hline
				      $(9)_{10}= (\dots)_2$  & $(10)_{10}=(\dots)_2$  & $(11)_{10}=(\dots)_2$  & $(12)_{10}=(\dots)_2$  \\
				      \hline
				      $(13)_{10}= (\dots)_2$ & $(14)_{10}= (\dots)_2$ & $(15)_{10}= (\dots)_2$ & $(16)_{10}= (\dots)_2$ \\
				      \hline
			      \end{tabularx}
			\item<2->{Combien faudra-t-il de chiffres en base 2 pour écrire $32$ ? \\
			      \lpo[1]}

		\end{itemize}
	\end{exampleblock}
\end{frame}

% Autre base
\begin{frame}
	\mframe{\Encodage}
	\begin{block}{Autre base}
		\begin{itemize}
			\item<1-> Nous savons écrire les entiers naturels en base \textcolor{red}{10} en utilisant \textcolor{red}{10} chiffres, chaque chiffre étant multiplié par une puissance de \textcolor{red}{10}.
			\item<2-> Nous savons écrire les entiers naturels en base \textcolor{red}{2} en utilisant \textcolor{red}{2} chiffres,chaque chiffre étant multiplié par une puissance de \textcolor{red}{2}.
			\item<3-> On montre qu'il est en fait possible, pour tout entier \textcolor{red}{$b \geq 2$} d'écrire les entiers naturels dans la base \textcolor{red}{$b$} en utilisant \textcolor{red}{$b$} chiffres. Chaque chiffre sera alors multiplié par une puissance de \textcolor{red}{$b$}.
		\end{itemize}
	\end{block}
	\begin{exampleblock}{Un exemple en base 5}
		\onslide<4->{
			\begin{tabular}{lcl}
				$\base{421}{5}$ & $=$ & $4 \times 5^2 + 2 \times 5^1 + 1 \times 5^0$ \\
				$\base{421}{5}$ & $=$ & $\base{111}{10}$                             \\
			\end{tabular}\\}
		\onslide<6->{Attention, les chiffres en base 5 sont 0, 1, 2, 3 et 4. Par conséquent écrire $\base{67}{5}$ n'a pas de sens !}
	\end{exampleblock}
\end{frame}


% Ecriture hexadécimale
\begin{frame}
	\mframe{\Encodage}
	\begin{alertblock}{La base 16 : écriture hexadécimale}
		\begin{itemize}
			\item<1-> En informatique, outre la base 2, on utilise aussi beaucoup la base 16.
			\item<2-> En base 16, il y a 16 chiffres : $0,1,2,3,4,5,6,7,8,9$ et $A,B,C,D,E,F$ (n'ayant plus de \og chiffres habituels \fg, on a utilisé les lettres de l'alphabet comme chiffres manquants)
			\item<3-> Comme 16 est une puissance de 2 ($16=2^4$),  on peut aisément passer de l'écriture binaire à l'écriture hexadécimale en regroupant les chiffres en base 2 par groupe de 4. :\\
		\end{itemize}
	\end{alertblock}
\end{frame}


\begin{frame}
	\mframe{\Encodage}
	\begin{block}{Conversion}
		\renewcommand{\arraystretch}{0.8}
		\begin{center}
			\begin{tabularx}{0.4\textwidth}{|X|X|X|}
				\hline
				hex. & bin.  & dec.  \\
				\hline
				0    & 0000  & 0     \\
				1    & 0001  & 1     \\
				2    & 0010  & 2     \\
				3    & 0011  & 3     \\
				4    & 0100  & 4     \\
				5    & 0101  & 5     \\
				6    & \dots & \dots \\
				7    & \dots & \dots \\
				8    & \dots & \dots \\
				9    & \dots & \dots \\
				A    & \dots & \dots \\
				B    & \dots & \dots \\
				C    & \dots & \dots \\
				D    & \dots & \dots \\
				E    & \dots & \dots \\
				F    & \dots & \dots \\
				\hline
			\end{tabularx}
		\end{center}
	\end{block}
\end{frame}

\begin{frame}
	\mframe{\Encodage}
	\begin{block}{Conversion}
		\renewcommand{\arraystretch}{0.8}
		\begin{center}
			\begin{tabularx}{0.4\textwidth}{|X|X|X|}
				\hline
				hex. & bin. & dec. \\
				\hline
				0    & 0000 & 0    \\
				1    & 0001 & 1    \\
				2    & 0010 & 2    \\
				3    & 0011 & 3    \\
				4    & 0100 & 4    \\
				5    & 0101 & 5    \\
				6    & 0110 & 6    \\
				7    & 0111 & 7    \\
				8    & 1000 & 8    \\
				9    & 1001 & 9    \\
				A    & 1010 & 10   \\
				B    & 1011 & 11   \\
				C    & 1100 & 12   \\
				D    & 1101 & 13   \\
				E    & 1110 & 14   \\
				F    & 1111 & 15   \\
				\hline
			\end{tabularx}
		\end{center}
	\end{block}
\end{frame}

%Question Flash
\begin{frame}
	\mframe{\Encodage}
	\begin{exampleblock}{\textcolor{yellow}{\flash} {Question flash}}
		\begin{itemize}
			\item<1-> Ecrire $\base{3EA}{16}$ en base 10
			\item<2-> Ecrire $\base{3EA}{16}$ en base 2
			\item<3-> Ecrire $\base{1101001011}{2}$ en base 16
			\item<4-> Ecrire $\base{1101001011}{2}$ en base 10
		\end{itemize}
	\end{exampleblock}
\end{frame}

%Algorithme des divisions successives
\begin{frame}
	\mframe{\Encodage}
	\begin{block}{\textcolor{yellow}{\flash} {Algorithme des divisions successives}}
		\begin{itemize}
			\onslide<1->{\item L'algorithme des \textcolor{blue}{divisions successives}, permet d'écrire un nombre donnée en base 10 dans n'importe quelle base $b$. Le principe est d'effectuer les divisions euclidiennes successives par $b$, les restes de ces divisions sont les chiffres du nombre dans la base $b$.}
			      \onslide<2->{\item Pour écrire $N$ en base $b$ :}
			      \begin{enumerate}
				      \item<3-> Faire la division euclidienne de $N$ par $b$, soit $Q$ le quotient et $R$ le reste. \\
				            (c'est à dire écrire $N = Q\times b + R$ avec $R<b$)
				      \item<4-> Ajouter $R$ aux chiffres de $N$ en base $b$
				      \item<5-> Si $Q=0$ s'arrêter, sinon recommencer à partir de l'étape 1 en remplaçant $N$ par $Q$.
			      \end{enumerate}
		\end{itemize}
	\end{block}
\end{frame}

%Exemple algorithme des divisions successives
\begin{frame}
	\mframe{\Encodage}
	\begin{exampleblock}{Exemple d'utilisation de l'algorithme des divisions successives}
		Donner l'écriture en base 16 de $\base{2019}{10}$. \\ \pause
		\begin{tabular}{lllllll}
			$2019$                                & $=$               & $\onslide<3->{\textcolor{blue}{126}}$ & $\times$               & $16$               & $+$               & $\onslide<4->{\textcolor{red}{\boxed{3}}} $  \\
			$\onslide<5->{\textcolor{blue}{126}}$ & \onslide<5->{$=$} & $\onslide<6->{\textcolor{blue}{7}}$   & \onslide<5->{$\times$} & \onslide<5->{$16$} & \onslide<5->{$+$} & $\onslide<7->{\textcolor{red}{\boxed{14}}} $ \\
			$\onslide<8->{\textcolor{blue}{7}}$   & \onslide<8->{$=$} & $\onslide<9->{\textcolor{blue}{0}}$   & \onslide<8->{$\times$} & \onslide<8->{$16$} & \onslide<8->{$+$} & $\onslide<10->{\textcolor{red}{\boxed{7}}} $ \\
		\end{tabular} \\
		\onslide<14->{
			Le quotient est nul, l'algorithme s'arrête et les chiffres en base 16 sont les restes obtenus à chaque étape donc  $\base{2019}{10}=\base{7E3}{16}$ (car 14 correspond au chiffre E).}
	\end{exampleblock}
\end{frame}

%Exemple algorithme des divisions successives
\begin{frame}
	\mframe{\Encodage}
	\begin{exampleblock}{Exemple d'utilisation de l'algorithme des divisions successives}
		Donner l'écriture en base 16 de $\base{9787}{10}$. \\ \pause
		\begin{tabular}{lllllll}
			$9787$                                & $=$                 & $\onslide<3->{\textcolor{blue}{611}}$ & $\times$                & $16$                & $+$                & $\onslide<4->{\textcolor{red}{\boxed{11}}} $ \\
			$\onslide<5->{\textcolor{blue}{611}}$ & \onslide<5->{$=$}   & $\onslide<6->{\textcolor{blue}{38}}$  & \onslide<5->{$\times$}  & \onslide<5->{$16$}  & \onslide<5->{$+$}  & $\onslide<7->{\textcolor{red}{\boxed{3}}} $  \\
			$\onslide<8->{\textcolor{blue}{38}}$  & \onslide<8->{$=$}   & $\onslide<9->{\textcolor{blue}{2}}$   & \onslide<8->{$\times$}  & \onslide<8->{$16$}  & \onslide<8->{$+$}  & $\onslide<10->{\textcolor{red}{\boxed{6}}} $ \\
			$\onslide<11->{\textcolor{blue}{2}}$  & \onslide<11->{ $=$} & $\onslide<12->{\textcolor{blue}{0}}$  & \onslide<11->{$\times$} & \onslide<11->{$16$} & \onslide<11->{$+$} & $\onslide<13->{\textcolor{red}{\boxed{2}}} $ \\
		\end{tabular} \\
		\onslide<14->{
			Le quotient est nul, l'algorithme s'arrête et les chiffres en base 16 sont les restes obtenus à chaque étape donc  $\base{9781}{10}=\base{263B}{16}$ (car 11 correspond au chiffre B).}
	\end{exampleblock}
\end{frame}


%Exemple algorithme des divisions successives (base 2)
\begin{frame}
	\mframe{\Encodage}
	\begin{exampleblock}{Exemple d'utilisation de l'algorithme des divisions successives}
		Donner l'écriture en base 2 de $\base{786}{10}$. \\ \pause
		\begin{tabular}{lllllll}
			$786$                                 & $=$                 & $\onslide<3->{\textcolor{blue}{393}}$ & $\times$                & $2$                & $+$                & $\onslide<4->{\textcolor{red}{\boxed{0}}} $  \\
			$\onslide<5->{\textcolor{blue}{393}}$ & \onslide<5->{$=$}   & $\onslide<6->{\textcolor{blue}{196}}$ & \onslide<5->{$\times$}  & \onslide<5->{$2$}  & \onslide<5->{$+$}  & $\onslide<7->{\textcolor{red}{\boxed{1}}} $  \\
			$\onslide<8->{\textcolor{blue}{196}}$ & \onslide<8->{$=$}   & $\onslide<9->{\textcolor{blue}{98}}$  & \onslide<8->{$\times$}  & \onslide<8->{$2$}  & \onslide<8->{$+$}  & $\onslide<10->{\textcolor{red}{\boxed{0}}} $ \\
			$\onslide<11->{\textcolor{blue}{98}}$ & \onslide<11->{ $=$} & $\onslide<12->{\textcolor{blue}{49}}$ & \onslide<11->{$\times$} & \onslide<11->{$2$} & \onslide<11->{$+$} & $\onslide<13->{\textcolor{red}{\boxed{0}}} $ \\
			$\onslide<14->{\textcolor{blue}{49}}$ & \onslide<14->{ $=$} & $\onslide<15->{\textcolor{blue}{24}}$ & \onslide<14->{$\times$} & \onslide<14->{$2$} & \onslide<14->{$+$} & $\onslide<16->{\textcolor{red}{\boxed{1}}} $ \\
			$\onslide<17->{\textcolor{blue}{24}}$ & \onslide<17->{ $=$} & $\onslide<18->{\textcolor{blue}{12}}$ & \onslide<17->{$\times$} & \onslide<17->{$2$} & \onslide<17->{$+$} & $\onslide<19->{\textcolor{red}{\boxed{0}}} $ \\
			$\onslide<20->{\textcolor{blue}{12}}$ & \onslide<20->{ $=$} & $\onslide<21->{\textcolor{blue}{6}}$  & \onslide<20->{$\times$} & \onslide<20->{$2$} & \onslide<20->{$+$} & $\onslide<22->{\textcolor{red}{\boxed{0}}} $ \\
			$\onslide<23->{\textcolor{blue}{6}}$  & \onslide<23->{ $=$} & $\onslide<24->{\textcolor{blue}{3}}$  & \onslide<23->{$\times$} & \onslide<23->{$2$} & \onslide<23->{$+$} & $\onslide<25->{\textcolor{red}{\boxed{0}}} $ \\
			$\onslide<26->{\textcolor{blue}{3}}$  & \onslide<26->{ $=$} & $\onslide<27->{\textcolor{blue}{1}}$  & \onslide<26->{$\times$} & \onslide<26->{$2$} & \onslide<26->{$+$} & $\onslide<28->{\textcolor{red}{\boxed{1}}} $ \\
			$\onslide<29->{\textcolor{blue}{1}}$  & \onslide<29->{ $=$} & $\onslide<30->{\textcolor{blue}{0}}$  & \onslide<29->{$\times$} & \onslide<29->{$2$} & \onslide<29->{$+$} & $\onslide<31->{\textcolor{red}{\boxed{1}}} $ \\
		\end{tabular} \\
		\onslide<32->{
			Le quotient est nul, l'algorithme s'arrête et $\base{786}{10}=\base{1100010010}{2}$.}
	\end{exampleblock}
\end{frame}


%ASCII et Latin-1
\begin{frame}
	\mframe{\Encodage}
	\begin{block}{Représentation des caractères : code {\sc ascii}}
		\begin{itemize}
			\item<1-> Dès les années 1960, Le code \textcolor{blue}{\sc ascii} (American Standard Code for Information Interchange) a crée un standard pour la représentation des caractères.
			\item<2-> Ce code n'utilisait que 7 bits et donc ne pouvait représenter que 128 caractères.
			\item<3-> L'encodage \textcolor{blue}{Latin-1} (ou ISO-8859-1), a étendu le code {\sc ascii} à 8 bits (256 caractères représentables) en intégrant notamment les lettres latines accentuées.
		\end{itemize}
	\end{block}
\end{frame}

%ASCII et Latin-1
\begin{frame}
	\mframe{\Encodage}
	\begin{block}{Représentation des caractères : unicode}
		Le codage \textcolor{blue}{\sc utf-8} (Unicode Transformation Format) s'est imposé comme standard d'encodage des caractères.
		\begin{itemize}
			\item<1-> les caractères sont représentés sur un nombre variable d'octets (de 1 à 4)
			\item<2-> compatibilité avec {\sc ascii}
			\item<3-> possibilités de représenter plusieurs centaine de milliers de caractères
		\end{itemize}
	\end{block}
\end{frame}

%Exemples
\begin{frame}
	\mframe{\Encodage}
	\begin{exampleblock}{Exemples}
		\renewcommand{\arraystretch}{1.2}
		\begin{center}
			\begin{tabular}{|c|c|c|c|}
				\cline{2-4}
				\multicolumn{1}{c|}{} & {\sc ascii}               & {\sc Latin-1}             & {\sc utf-8}           \\
				\hline
				\alt<2->{A            & 65                        & 65                        & 65}{        &  &  & } \\
				\hline
				\alt<3->{À            & \textcolor{red}{\faTimes} & 192                       & 192}{       &  &  & } \\
				\hline
				\alt<4->{$\beta$      & \textcolor{red}{\faTimes} & \textcolor{red}{\faTimes} & 946}{       &  &  & } \\
				\hline
			\end{tabular}
		\end{center}
	\end{exampleblock}
\end{frame}

%Exemples
\begin{frame}
	\mframe{\Encodage}
	\begin{block}{Encodage en Python}
		En Python, 
		\begin{itemize}
			\item<2-> \pmc{chr}{\tt (code)} renvoie le caractère de code {\sc utf-8} {\tt code}
			\item<3->  \pmc{ord}{\tt (caractere)} renvoie le code  {\sc utf-8} du caractère {\tt caractere}
		 \end{itemize}
	\end{block}
	\onslide<4->{
	\begin{exampleblock}{Exemple}
		{\tt >}{\tt>}{\tt>}\pmc{chr}{\tt (946)} \\
		{\tt '}$\beta${\tt '} \\
		{\tt >}{\tt>}{\tt>}\pmc{ord}{\tt ('À')} \\
		{\tt 192}\\
	\end{exampleblock}}
\end{frame}

\end{document}