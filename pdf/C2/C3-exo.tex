\documentclass[11pt,a4paper]{article}

\usepackage{Act}
\usepackage{listings}
\begin{document}
\input{\detokenize{/home/fenarius/Travail/Cours/Commun/latex/Macros.tex}}

\newcommand{\ind}{\phantom{\ \ \ \ }}
% Idées : IMC, signe astrologique
%
%

\Exos{Ecriture binaire et hexadécimale des entiers positifs}{\Pre}

\Exo{Quelques conversions pour démarrer !}{\capacite}\\
Compléter le tableau de conversion suivant :
\renewcommand{\arraystretch}{1.5}
\begin{center}
\begin{tabularx}{0.9\textwidth}{|X|X|X|}
\hline
Ecriture décimale & Ecriture binaire & Ecriture héxadécimale \\
\hline
 $\base{201}{10}$ &  &  \\
 \hline
 & & $\base{EA}{16}$ \\
 \hline
 $\base{57}{10}$ &  &  \\
 \hline
  & $\base{00100001}{2}$ &  \\
 \hline

 $\base{128}{10}$  & &  \\
 \hline
 $\base{163}{10}$ & & \\
 \hline
  & & $\base{5B}{16} $\\
 \hline
 &  $\base{10010101}{2}$ & \\
  \hline
 & $\base{10010010}{2}$ &  \\
 \hline
\end{tabularx}
\end{center}
\aide \; Pour convertir depuis la base 10 dans la base 2 ou la base 16, utiliser si besoin, l'algorithme des divisions successives vu en cours. Refaire au préalable les exemples du cours si nécessaire.
\vspace{0.2cm}

\Exo{Un peu de réflexion}{\raisonnement}
\QListe
\item Quel est le plus grand nombre entier positif écrit en utilisant 10 chiffres en base 2 ?
\item Que peut-on dire d'un nombre dont l'écriture en base 2 ne contient qu'un seul chiffre 1 ?
\item \SQListe
\item L'écriture en base 2 d'un nombre divisible par 2 se termine forcément par quel chiffre ? Pourquoi ?
\item De façon générale, soit $b$ un entier supérieur ou égal à 2, que dire de l'écriture en base $b$ d'un nombre divisible par $b$ ?
\FinListe
\item En base 10, un million s'écrit avec 7 chiffres, combien en faut-il pour l'écrire en base 2 ?
\FinListe
\vspace{0.2cm}


\Exo{Enigme}{\raisonnement} \\
Il manque des chiffres (remplacées par des \_) dans le nombre binaire suivant : $\quad \_001\_\_111\_$ \\
Pouvez-vous retrouver les chiffres manquants, sachant qu'il n'y a pas de zéros non significatifs dans ce nombre, qu'il est divisible par 2, qu'il contient un nombre impair de zéro. Et enfin que son écriture décimale dépasse $\base{355}{10}$.
\vspace{0.2cm}

\Exo{Dessin mystère}{\ordinateur} 
\QListe
\item Donner dans chaque cas l'écriture binaire sur un octet du nombre indiqué, puis noircir les cases correspondates de la ligne du tableau situé à côté du nombre. Vous devez voir apparaître un dessin !\
\begin{tabularx}{0.64\textwidth}{p{1.3cm}X|p{0.2cm}|p{0.2cm}|p{0.2cm}|p{0.2cm}|p{0.2cm}|p{0.2cm}|p{0.2cm}|p{0.2cm}|}
\cline{3-10}
$\base{60}{10}=$ & $\left( \_\;\_\;\_\;\_\;\_\;\_\;\_\;\_\; \right)_{2} \qquad $ &  & & & & & & & \\
\cline{3-10}
$\base{66}{10}=$ & $\left( \_\;\_\;\_\;\_\;\_\;\_\;\_\;\_\; \right)_{2} \qquad $ &  & & & & & & & \\
\cline{3-10}
$\base{65}{10}=$ & $\left( \_\;\_\;\_\;\_\;\_\;\_\;\_\;\_\; \right)_{2} \qquad $ &  & & & & & & & \\
\cline{3-10}
$\base{129}{10}=$ & $\left( \_\;\_\;\_\;\_\;\_\;\_\;\_\;\_\; \right)_{2} \qquad $ &  & & & & & & & \\
\cline{3-10}
$\base{165}{10}=$ & $\left( \_\;\_\;\_\;\_\;\_\;\_\;\_\;\_\; \right)_{2} \qquad $ &  & & & & & & & \\
\cline{3-10}
$\base{153}{10}=$ & $\left( \_\;\_\;\_\;\_\;\_\;\_\;\_\;\_\; \right)_{2} \qquad $ &  & & & & & & & \\
\cline{3-10}
$\base{42}{16}=$ & $\left( \_\;\_\;\_\;\_\;\_\;\_\;\_\;\_\; \right)_{2} \qquad $ &  & & & & & & & \\
\cline{3-10}
$\base{3C}{16}=$ & $\left( \_\;\_\;\_\;\_\;\_\;\_\;\_\;\_\; \right)_{2} \qquad $ &  & & & & & & & \\
\cline{3-10}
\end{tabularx}
\item  Créer un fichier nommé \texttt{dessin.pbm}, dont la première ligne est \texttt{8 8}. Sur les lignes suivantes taper les écriture binaire ci-dessus. Ouvrez ce fichier avec un visualiseur d'image. Qu'en pensez-vous ?
\FinListe
\vspace{0.2cm}


\Exo{Fonction python}{\python \; \ordinateur}
\QListe
\item Tester la fonction \texttt{bin} de Python en affichant par exemple \texttt{bin(201)} et \texttt{bin(57)}. Comparer avec les réponses de l'exercice 1. Quelle est votre hypothèse sur \texttt{bin} ?
\item Confirmer cette hypothèse en faisant afficher l'aide de la fonction \texttt{bin}.
\item Reprendre les deux questions ci-dessus pour la fonction \texttt{hex}.
\FinListe
\vspace{0.2cm}

\end{document}