\PassOptionsToPackage{dvipsnames,table}{xcolor}
\documentclass[10pt]{beamer}
\usepackage{Cours}

\begin{document}

\input{\detokenize{/home/fenarius/Travail/Cours/NSIPremiere/docs/commun/MacrosCours.tex}}
\setcounter{numchap}{10}
\pythonmode
\newcommand{\BinaireNegatif}{\cnum Représentation des entiers négatifs}

% Complément à 2
\begin{frame}
\mframe{\BinaireNegatif}
\begin{alertblock}{Complément à 2}
Pour représenter un entier négatif en machine, on utilise la méthode du \textcolor{red}{complément à 2} :
\begin{enumerate}
\item<2-> on commence par écrire la représentation binaire de la valeur absolue de ce nombre
\item<3-> on inverse tous les bits de cette représentation 
\item<4-> on ajoute 1, sans tenir compte de la dernière retenue éventuelle
\end{enumerate}
\end{alertblock}
\begin{block}{Avantages de cette représentation}
\begin{itemize}
\item<5-> l'algorithme d'addition classique des nombres en base 2 fonctionne 
\item<6-> tous les nombres sont représentés de façon unique (pas de double représentation pour zéro).
\item<7-> Le bit le plus à gauche est le bit de signe, il vaut 1 lorsque le nombre est négatif, 0 sinon.
\end{itemize}
\end{block}
\end{frame}


% Exemples
\begin{frame}
\mframe{\BinaireNegatif}
\begin{exampleblock}{Exemples}
\begin{itemize}
\item Sur 8 bits, donner l'écriture en complement à 2 de $-12$
\begin{tabular}{ll}
\onslide<2->{\textbf{1.} On écrit $12=(8+4)$ en binaire sur 8 bits :} & \onslide<3->{$00001100$} \\
\onslide<4->{\textbf{2.} On inverser tous les bits :} & \onslide<5->{$11110011$} \\
\onslide<6->{\textbf{3.} On ajoute 1 :}  & \onslide<7->{$11110100$} \\
\end{tabular}
\item Même question pour $75$
\begin{tabular}{ll}
\onslide<7->{\textbf{1.} On écrit $75=64+8+2+1$ en binaire sur 8 bits :} & \onslide<8->{$01001011$} \\
\onslide<9->{\textbf{2.} On inverser tous les bits :} & \onslide<10->{$10110100$} \\
\onslide<11->{\textbf{3.} On ajoute 1 :}  & \onslide<12->{$10110101$} \\
\end{tabular}
\end{itemize}
\end{exampleblock}
\end{frame}
\end{document}
