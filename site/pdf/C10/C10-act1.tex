\documentclass[11pt,a4paper]{article}

\usepackage{Act}

\begin{document}
\input{\detokenize{/home/fenarius/Travail/Cours/NSIPremiere/docs/commun/Macros.tex}}
\ModeActivite
\Activites{\debase\; -- C10 : Représentation des entiers relatifs}{\Pre}


\begin{tcolorbox}[title=\textcolor{black}{\rappel \; Ecriture binaire d'un entier positif},colbacktitle=lightgray]
\begin{itemize}
\item[$\bullet$] Passage du binaire au décimal :\\
Pour écrire $(1011010)_2$ en base 10, puisque chaque chiffre correspond à une puissance de 2 :\\
\begin{tabular}{|p{0.4cm}|p{0.4cm}|p{0.4cm}|p{0.4cm}|p{0.4cm}|p{0.4cm}|p{0.4cm}|l}
\cline{1-7}
${2^6}$ & ${2^5}$ & ${2^4}$ & ${2^3}$ & ${2^2}$ & ${2^1}$ & ${2^0}$ \\ 
\cline{1-7}
 $1$ &  $0$ &  $1$ &  $1$ &  $0$ &  $1$ &  $0$  & = $ 1 \times$ $2^6$+ $ 1 \times 2^4$+$1\times 2^3$ + $1\times 2^1$ \\
 \cline{1-7}
\multicolumn{7}{l}{}& $= 64 + 16 + 8 + 2 = 90$ \\ 
\end{tabular}

\item[$\bullet$] Passage du décimal au binaire :\\
Sur des petits exemples et en faisant du calcul mental (bien connaître les premières puissances de 2) on peut retrouver directement le résultat. Par exemple : \\
$41 = 32  + 8 + 1$ donc $(41)_{10}=(101001)_2$ \\
Sinon, on utilise l'algorithme des divisions successives, la suite des restes est l'écriture du nombre. Par exemple pour $107$ : \\
\begin{tabular}{lllllll}
$107$ & $=$ & $2$ & $\times$ & $53$ & $+$ & $\boxed{1}$ \\
$53$ & $=$ & $2$ & $\times$ & $26$ & $+$ & $\boxed{1}$ \\
$26$ & $=$ & $2$ & $\times$ & $13$ & $+$ & $\boxed{0}$ \\
$13$ & $=$ & $2$ & $\times$ & $6$ & $+$ & $\boxed{1}$ \\
$6$ & $=$ & $2$ & $\times$ & $3$ & $+$ & $\boxed{0}$ \\
$3$ & $=$ & $2$ & $\times$ & $1$ & $+$ & $\boxed{1}$ \\
$1$ & $=$ & $2$ & $\times$ & $0$ & $+$ & $\boxed{1}$ \\
\end{tabular}\\
Donc, $(107)_{10} = (1101011)_2$.
\end{itemize}
\end{tcolorbox}


%Nom de la première activité
\begin{Exercise}[title={Compter avec des 0 et des 1}]
% Rappel écriture binaire


\Question Quelques rappels sur la représentations des nombres positifs
\subQuestion Donner l'écriture binaire de 11. 
\subQuestion Donner l'écriture binaire de 93.
\subQuestion Poser et effectuer l'addition binaire de 11 et de 93.
\subQuestion Vérifier en l'écrivant en décimal que le résultat obtenu est bien 104.
\Question Cas des nombres négatifs : approche naïve. \\
Pour représenter les nombres négatifs en binaire, une première idée consiste à indiquer sur le bit le plus à gauche le signe du nombre : 0 si le nombre est positif et 1 si le nombre est négatif. Par exemple si on dispose d'un octet, le 8\textsuperscript{e} bit indique le signe et par exemple $(\textbf{1}0011010)_2$ est un nombre négatif, sa valeur absolue est $(0011010)_2=26$. En conclusion, avec cette représentation : $(10011010)_2 = (-26)_{10}$.
\subQuestion Donner l'écriture binaire de $-11$ avec cette représentation. Même question pour $-48$.
\subQuestion Donner la  valeur décimale de $(10111000)_2$ et celle de $(11100001)_2$.
\subQuestion Donner la  valeur décimale de $(10000000)_2$ et celle de $(00000000)_2$.
\subQuestion Poser et effectuer l'addition binaire de $11$ et de $-11$. Obtient-on le résultat attendu ?
\Question Complément à 2 \\
Afin de palier aux problèmes de la représentation précédente, on utilise pour représenter les nombres négatifs en binaire la méthode dite du \textit{complément à 2}. Pour représenter un nombre négatif par exemple $-42$ :
\begin{enumerate}
\item on commence par écrire la valeur absolue du nombre en binaire. Ici $42 = 32 + 8  + 2$ donc \\$42 = (00101010)_2$
\item on inverse tous les bits  $00101010 \longrightarrow 11010101$, c'est cette inversion des bits qui donne son nom à la méthode.
\item on fait l'addition binaire de 1 au nombre obtenu : $(11010101)_2 + (1)_2 = (11010110)$. Attention lors de cette opération on ne tient pas compte de la dernière retenue.
\end{enumerate}
L'intérêt de cette méthode est d'éliminer les inconvénients de la technique précédente. Nous allons le vérifier.
\subQuestion Donner l'écriture en complément à 2 de $11$. Vérifier que l'addition binaire à $11$ donne bien $0$.
\subQuestion Même question pour $93$.
\subQuestion Donner l'écriture en complément à 2 de $-128$.
\subQuestion Conclure
\end{Exercise}
\end{document}