\documentclass[11pt,a4paper]{article}

\usepackage{Act}

\begin{document}
\input{\detokenize{/home/fenarius/Travail/Cours/NSIPremiere/docs/commun/Macros.tex}}
\ModeActivite
\Activites{\debase\; -- C2 : Représentation des entiers et des caractères}{\Pre}

%Nom de la première activité

\begin{Exercise}[title={Codage hexadécimal},number=2]

	Comme vu en cours, il faut jusqu'à 8 chiffres en base 2 (8 bits) pour écrire les entiers de 0 à 255, le codage \textit{hexadécimal} est bien plus utilisé car plus compact. En effet,  deux caractères suffisent à écrire un nombre compris entre 0 et 255.
	\begin{enumerate}
		\item[] Etape 1 : séparer les 8 bits en deux groupes de 4, par exemple $ 11001111 \rightarrow 1100 \ 1111 $
		\item[] Etape 2 : consulter le tableau suivant et remplacer chaque groupe de 4 par le caractère correspondant, par exemple $ 11001111 \rightarrow CF $ \par \smallskip
		      \begin{tabular}{|c|c||c|c||c|c||c|c|}
			      \hline
			      Groupe & Caractère & Groupe & Caractère & Groupe & Caractère & Groupe & Caractère \\
			      \hline
			      0000   & 0         & 0001   & 1         & 0010   & 2         & 0011   & 3         \\
			      \hline
			      0100   & 4         & 0101   & 5         & 0110   & 6         & 0111   & 7         \\
			      \hline
			      1000   & 8         & 1001   & 9         & 1010   & A         & 1011   & B         \\
			      \hline
			      1100   & C         & 1101   & D         & 1110   & E         & 1111   & F         \\
			      \hline
		      \end{tabular}
	\end{enumerate}
    Le codage hexadécimal $CF$ est bien plus facile à retenir que le codage binaire $11001111$.
	\Question Appliquer la méthode
	\subQuestion Ecrire le codage hexadécimal de $10111001$.
	\subQuestion Ecrire $100_{10}$ en base 2 puis donner son codage hexadécimal.
	\Question Compter en base 16
    \subQuestion Dans le tableau de correspondance entre le groupe et le caractère, calculer la valeur décimale de chacun des groupes de 4 bits, que remarquez-vous ?
    \subQuestion Quelles devraient être les valeurs associées aux lettres $A$, $B$, $C$, $D$, $E$ et $F$ ?
    \subQuestion Si on considère chaque caractère du tableau comme un chiffre, dans quelle base travaille-t-on ? Justifier.
    \subQuestion Compléter : \\
    $ 187 = 12 \times 16 + \dots$, or \\
	$12 \rightarrow \dots$ et $11 \rightarrow  \dots$ donc \\
	$187_{10} = \dots \dots_{16}$ 
    \subQuestion Ecrire $162_{10}$ en base 16 \\
	\aide \; Effectuer la \textit{division euclidienne} de 162 par 16 de façon à écrire 162 sous la forme : \\$162 = q\times 16 + r$
    \subQuestion Même question pour $218_{10}$
    \Question de la base 16 vers la base 10
    \subQuestion Ecrire $6C_{16}$ en base 10
    \subQuestion Même question pour $17_{16}$
    \Question Le codage des couleurs \par
    En informatique, une couleur est représentée par trois valeurs pouvant aller de 0 à 255 . Ces valeurs représentent les niveaux des trois couleurs primaires rouge, vert et bleu. Une couleur est donc représentée sous la forme $(r,v,b)$ où $r$,$v$ et $b$ sont des entiers entre 0 et 255. Le codage hexadécimal (utilisé notamment en langage {\sc html}) représente une couleur sous la forme de six chiffres en base 16.
    \subQuestion Quelles sont les valeurs de rouge, vert et bleu dans le code $A1C077$ ?
    \subQuestion Expliquer pourquoi le code $FF0000$ représente le rouge pur.
    \subQuestion Expliquer pourquoi le code $808080$ représente le gris.
	\subQuestion Donner le code couleur {\sc html} de la couleur représentée en code $(r,v,b)$ par (28,140,212).
	\subQuestion Donner le code (r,v,b) de la couleur $18E54E$

\end{Exercise}
\end{document}