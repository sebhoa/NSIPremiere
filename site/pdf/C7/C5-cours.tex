\PassOptionsToPackage{dvipsnames,table}{xcolor}
\documentclass[10pt]{beamer}
\usepackage{Cours}

\begin{document}

\input{\detokenize{/home/fenarius/Travail/Cours/NSIPremiere/docs/commun/MacrosCours.tex}}
\setcounter{numchap}{7}

\pythonmode
\newcommand{\DB}{\cnum Lecture et traitement de données en tables}


% Retour données en table
\begin{frame}
	\mframe{\DB}
	\begin{alertblock}{Données en tables}
		\begin{itemize}
			\item<1-> Le traitement et l'analyse de données volumineuses (\textit{big data}) est l'une des activités principales en informatique de nos jours .
			\item<2-> Ces données sont souvent organisées en \textcolor{red}{tables}.
			\item<3-> Une ligne de données en table s'appelle un \textcolor{blue}{enregistrement}
			\item<4-> Une colonne s'apelle un \textcolor{blue}{champ}
			\item<5-> Les titres des colonnes sont les \textcolor{blue}{descripteurs}
		\end{itemize}
	\end{alertblock}
	\onslide<6->{
		\begin{block}{Format csv}
			Le format de fichier \textcolor{red}{csv} (\textit{\textcolor{red}{c}omma \textcolor{red}{s}eparated \textcolor{red}{v}alue}) représente  des données en tables. Chaque ligne du fichier est une donnée et sur chaque ligne les champs sont séparées par des virgules (ou parfois un autre caractère comme le point-virgule).
		\end{block}}
\end{frame}

% Exemple table et CSV
\begin{frame}
	\mframe{\DB}
	\begin{exampleblock}{Exemple}
		\begin{tabularx}{\textwidth}{Xp{1cm}X}
			$\bullet$\ Des données en table &  & $\bullet$ Représentation en fichier csv \\
			\begin{tabular}{|c|c|c|}
				\hline
				Nom      & Prénom & Naissance \\
				\hline
				Pascal   & Blaise & 1623      \\
				\hline
				Lovelace & Ada    & 1815      \\
				\hline
				Boole    & George & 1815      \\
				\hline
			\end{tabular}       &  &
			\begin{tabular}{|l|}
				\hline
				{\tt Nom;Prénom;Naissance} \\
				{\tt Pascal;Blaise;1623}   \\
				{\tt Lovelace;Ada;1815}    \\
				{\tt Boole;George;1815}    \\
				\hline
			\end{tabular}                                                    \\
		\end{tabularx}
		Le fichier csv à droite sera utilisé par la suite, on l'appelle \textcolor{OliveGreen}{\tt exemple.csv} de façon à y faire référence. \\
	\end{exampleblock}
	\onslide<2->{
	\begin{block}{Remarques}
		\begin{itemize}
			\item<2-> La première ligne du fichier csv décrit les champs, il contient les \textcolor{blue}{attributs} (appelés aussi \textcolor{blue}{descripteur}).
			\item<3-> Les données d'un fichier csv sont au format texte, par conséquent même une donnée numérique (comme ici l'année de naissance) est en fait une chaine de caractères.
		\end{itemize}}
	\end{block}
\end{frame}

% Module csv de Python
\begin{frame}
	\mframe{\DB}
	\begin{block}{Python et les fichiers csv}
		\begin{itemize}
			\item<1-> Le module \textcolor{blue}{\tt csv} de Python permet de récupérer les informations d'un fichier csv, sous forme de listes de listes ou de dictionnaires
			\item<2-> Pour les dictionnaires, ce sont alors les  descripteurs qui servent de clés.
			\item<3-> Tous les champs (même ceux contenant des nombres) sont récupérés sous forme de chaines de caractères (type \textcolor{blue}{\tt str} de Python) à la façon de ce qui se passe lors d'un {\tt input}. Faire donc attention lors de calculs ou de comparaisons avec les données de ces champs.
		\end{itemize}
	\end{block}
\end{frame}

% Exemple csv Python
\begin{frame}[fragile]
	\mframe{\DB}
	\begin{exampleblock}{Exemple}
		Récupération des éléments du fichier {\tt exemple.csv} ci-dessus dans un dictionnaire :
		\begin{lstlisting}	
	import csv
	fic=open("exemple.csv","r",encoding="utf-8")
	# Lecture sous forme de dictionnaire 
	donnees = list(csv.DictReader(fic,delimiter=';'))
	fic.close()
	\end{lstlisting}
		\onslide<2->{Après execution, on a par exemple \\ {\tt donnees[0]=\{'Nom' : 'Pascal', 'Prenom' : 'Blaise', 'Naissance' : '1623'\}} \\}
		\onslide<3->{C'est à dire que chaque ligne de la table correspond à un dictionnaire}
	\end{exampleblock}
\end{frame}

% Traitement des données en Python
\begin{frame}[fragile]
	\mframe{\DB}
	\begin{block}{Traitement des données}
		\begin{itemize}
			\item<1-> Une fois les données {\tt csv} lues et récupérées dans un dictionnaire, on peut trier les informations et y faire des recherches.
			\item<2-> Par exemple pour le fichier csv donné en exemple :
			      \begin{tabular}{|l|}
				      \hline
				      {\tt Nom;Prénom;Naissance} \\
				      {\tt Pascal;Blaise;1623}   \\
				      {\tt Lovelace;Ada;1815}    \\
				      {\tt Boole;George;1815}    \\
				      \hline
			      \end{tabular}
			\item<3-> Si les données sont récupérées dans la liste de dictionnaire {\tt personnages}. On peut afficher les personnes nées en 1815 avec :
			      \begin{lstlisting}
    	for p in personnages:
    		if p["Naissance"]=="1815":
    			print(p["Nom],p["Prénom"])
    \end{lstlisting}
		\end{itemize}
	\end{block}
\end{frame}

% Trier des données
\begin{frame}[fragile]
	\mframe{\DB}
	\begin{alertblock}{Trier une liste en Python}
		\begin{itemize}
			\item<1-> La fonction \textcolor{red}{\tt sorted} de Python permet de trier une liste. Elle renvoie la liste triée. La syntaxe est la suivante : \textcolor{red}{\tt liste\_triee = sorted(liste)}.
			\item<2-> Par exemple:
			      \begin{lstlisting}
	notes = [15,11,10,18,9]
	note_triees=sort(notes)
	print(notes_triees)
	 \end{lstlisting}
			      affichera : {\tt [9,10,11,15,18]}
			\item<3-> On peut obtenir un tri par ordre décroissant en indiquant {\tt reverse=True} \textcolor{red}{\tt liste\_triee = sorted(liste,reverse=True)}.
		\end{itemize}
	\end{alertblock}
\end{frame}

% Trier des données
\begin{frame}[fragile]
	\mframe{\DB}
	\begin{alertblock}{Trier une liste de dictionnaires}
		\begin{itemize}
			\item<1-> La fonction \textcolor{red}{\tt sorted} permet aussi de trier des listes de dictionnaires on indique alors le critère de tri à l'aide de l'option \textcolor{blue}{\tt key}.
			\item<2-> Par exemple:
			      \begin{lstlisting}
	def note(eleve):
		return eleve["Note"]	
	
	notes = [{"Prenom":"Albert","Note":15},{"Prenom":"Jim","Note":10},{"Prenom":"Sarah","Note":19}]
	notes.sort(key=note,reverse=True)
	print(notes)
	 \end{lstlisting}
			      affichera : {\tt [\{'Prenom': 'Sarah', 'Note': 19\}, \{'Prenom': 'Albert', 'Note': 15\}, \{'Prenom': 'Jim', 'Note': 10\}]}
		\end{itemize}
	\end{alertblock}
\end{frame}

\end{document}