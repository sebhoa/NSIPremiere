\PassOptionsToPackage{dvipsnames,table}{xcolor}
\documentclass[10pt]{beamer}
\usepackage{Cours}

\begin{document}

\input{\detokenize{/home/fenarius/Travail/Cours/NSIPremiere/docs/commun/MacrosCours.tex}}
\setcounter{numchap}{7}

\newcommand{\Arch}{\cnum Architecture des ordinateurs}


% Définition langage de programmation
\begin{frame}
	\mframe{\Arch}
	\begin{alertblock}{Modèle de Von Neumann}
		\begin{itemize}
			\item<1-> Les ordinateurs modernes sont construits autour d'un modèle défini par le mathématicien John Von Neumann en 1945 et appelé \textcolor{blue}{Architecture de Von Neumann}.
			\item<2-> Dans ce modèle, l'ordinateur se décompose en 5 parties distinctes :
			      \begin{enumerate}
				      \item<3-> Les dispositifs d'\textcolor{blue}{entrée} des données (ex : clavier, souris, écran tactile, réseau \dots),
				      \item<4-> La \textcolor{blue}{mémoire} qui stocke les données et les programmes (ex : mémoire cache, {\sc ram}, \dots)
				      \item<5-> L'\textcolor{blue}{unité arithmétique et logique {\sc ual}} qui effectue les opérations (addition, soustraction, comparaison, \dots) sur les données.
				      \item<6-> L'\textcolor{blue}{unité de contrôle} qui est chargé de la gestion de l'ordre des opérations (séquençage)
				      \item<7-> Les dispositifs de \textcolor{blue}{sortie} des données (ex : écran, imprimante, \dots)
			      \end{enumerate}
		\end{itemize}
	\end{alertblock}
\end{frame}

%Remarque architectecture von neumann
\begin{frame}
	\mframe{\Arch}
	\begin{block}{Remarques :}
		\begin{itemize}
			\item<1-> Dans les ordinateurs modernes, l'{\sc ual} et l'unité de contrôle sont regroupés dans le processeur ({\sc cpu} pour Central Processing Unit en anglais)
			\item<2-> Certains periphériques sont à la fois des dispositifs d'entrée et de sortie. Par exemple, le disque dur car on peut y lire (entrée) et écrire (sortie) des données.
			\item<3-> Par rapport au modèle initial, les ordinateurs actuels possèdent parfois plusieurs processeurs ou coeurs.
		\end{itemize}
	\end{block}
\end{frame}


%Schéma
\begin{frame}
	\mframe{\Arch}
	\setlength{\shadowsize}{1pt}
	\psset{linewidth=0.7pt}
	\begin{block}{Schéma représentant l'architecture de Von Neumann :}
		\begin{tabularx}{0.9\textwidth}{Xp{1cm}|Y|p{1cm}X}
			\cline{3-3}
			                                                                                                                                                                                                                                                                                                                                                                                                                                                                                                                                              &  &                                                  &  & \\
			                                                                                                                                                                                                                                                                                                                                                                                                                                                                                                                                              &  & \textcolor{blue}{\textbf \faMicrochip \, {\Large \sc cpu}} &  & \\
			\onslide<3->{\rnode{ram}{\psshadowbox{\makebox[2cm]{\par\noindent\rule[-1.4cm]{0pt}{3cm} \textcolor{red}{\textbf{\faMemory \ Mémoire}} }} } }                                                                                                                                                                                                                                                                                                                                                                                                 &  &
			\onslide<2->{\rnode{ual}{\psframebox[framearc=.3,framesep=0,linecolor=Sepia,linewidth=1pt]{\psframebox*[framearc=.3,fillcolor=grispale]{\parbox[c][1.5cm][c]{1.8cm}{\textcolor{Sepia}{\large \  \ \ \sc ual} \newline {\small Unité arithmétique et logique}}}}} \newline \rule{0pt}{0.8cm} \newline {\rnode{uc}{\psframebox[framearc=.3,framesep=0,linecolor=Sepia,linewidth=1pt]{\psframebox*[framearc=.3,fillcolor=grispale]{\parbox[c][1.5cm][c]{1.8cm}{\textcolor{Sepia}{\large  \ \  \ \sc uc} \newline {\small Unité de contrôle}}}}} }   } &  &
			\onslide<4->{\rnode{in}{\shadowbox{\makebox[2cm]{\par\noindent\rule[-0.4cm]{0pt}{1cm} {\textcolor{red}{ \faArrowAltCircleLeft \; Entrées}} }}} \newline \rule{0pt}{0.8cm} {\rnode{out}{\shadowbox{\makebox[2cm]{\par\noindent\rule[-0.4cm]{0pt}{1cm} \textcolor{red}{ Sorties \; \faArrowAltCircleRight} }}} }   }                                                                                                                                                                                                                                                                                       \\
			                                                                                                                                                                                                                                                                                                                                                                                                                                                                                                                                              &  &                                                  &  & \\
			\cline{3-3}
		\end{tabularx}
		\onslide<5->{\ncline[ linecolor=blue, linewidth=2px, offsetA=0.3cm]{<->}{ram}{ual}}
		\onslide<6->{\ncline[linewidth=2px, linecolor=blue, offsetA=-0.7cm,nodesepA=0.3cm]{<->}{ram}{uc}}
		\onslide<7->{\ncline[linecolor=blue,linewidth=2px]{<->}{ual}{uc}}
		\onslide<8->{\ncline[linecolor=blue,linewidth=2px]{->}{in}{ual}}
		\onslide<9->{\ncline[linecolor=blue,offsetA=0.2cm,offsetB=-0.2cm,nodesepA=-0.1cm,nodesepB=-0.1cm,linewidth=2px]{->}{ual}{out}}
	\end{block}
\end{frame}


% Transistor et booléens
\begin{frame}
	\mframe{\Arch}
	\begin{block}{Remarques :}
		\begin{itemize}
			\item<1-> Le composant de base des ordinateurs est le \textit{transistor}, un composant électronique ne pouvant être que dans deux états. Soit il laisse passer le courant (état \textcolor{red}{1}), soit il ne le laisse pas passer (état \textcolor{red}{0}).
			\item<2-> Toutes les données représentées dans un ordinateur le sont donc sous forme de 0 et de 1.
			\item<3-> Dès les années 1850, dans des travaux sur la logique, le mathématicien britannique Georges Boole avait travaillé sur des variables ne pouvant prendre que deux valeurs 0  ou 1.
			\item<4-> On appelle, ces variables des \textcolor{red}{booléens}. On définit trois opérations de base que nous allons détailler sur les booléens : le \textcolor{red}{non}, le \textcolor{red}{et} et le \textcolor{red}{ou}.
		\end{itemize}
	\end{block}
\end{frame}





\end{document}