\documentclass[11pt,a4paper]{article}

\usepackage{Act}
\usepackage{listings}

\begin{document}
\input{\detokenize{/home/fenarius/Travail/Cours/Commun/latex/Macros.tex}}

\DNSI{Systèmes d'exploitation}{\Pre}\vspace{0.2cm}

\bashmode



\Exo{Gestion des dossiers}{5 pts}\\
En tapant la commande {\tt tree} dans son répertoire personnel (c'est à dire dans {\tt /home/alfred/}), un utilisateur d'un système Linux a obtenu le résultat ci-dessous. \\
\tei{arborescence.eps}{0.7}{1}{On suppose que le répertoire courant est {\tt /home/alfred/}.
\QListe
\item Ecrire une commande permettant de se déplacer dans le dossier {\tt Factures} en donnant un chemin relatif\\
\lpo[1]
\item Ecrire une commande permettant d'y creer les dossiers {\tt Eau} et {\tt Electricité}\\
\lpo[1]
\item Alfred a commis une erreur en nommant l'un des dossiers du répertoire {\tt Photos}, c'est {\tt 2020} et non {\tt 20020}. Ecrire une ou plusieurs commandes permettant de changer le nom de ce répertoire.\\
\lpo[2]
\FinListe
}
\QListe[1]
\item Donner une commande permettant de se rendre dans le dossier {\tt Images} en donnant un chemin absolu.\\
\aide\; on rappelle que l'arborescence ci-dessus est celle de {\tt /home/alfred/}\\
\lpo[1]
\item Lorsqu'on se trouve dans le dossier {\tt Images}, donner une commande permettant de se rendre dans le dossier {\tt Importants} en donnant un chemin relatif : \\
\lpo[1]
\FinListe
\vspace{0.3 cm}

\Exo{Permissions sur les fichiers}{5 pts}
\QListe
\item Dans chaque cas écrire une commande permettant de modifier les droits sur le fichier \texttt{monfichier} de la façon indiquée :
\SQListe
\item Supprimer le droit de lecture pour le groupe et pour les autres\\
\lpo[1]
\item Supprimer le droit d'écriture pour l'utilisateur \\
\lpo[1]
\item Ajouter le droit d'écriture et de lecture pour le propriétaire et le groupe\\
\lpo[1]
\item Ajouter le droit d'éxécution à tout le monde \\
\lpo[1]
\FinListe
\item Rappeler le nom de la commande permettant d'afficher le contenu d'un fichier à l'écran. Quel sera l'effet de cette commande si on ne possède pas les droits de lecture sur le fichier ?\\
\lpo[2]
\FinListe
\vspace{0.3cm}

\Exo{Pour le fun ...}{0 pt}\\
Vous pouvez utiliser Webminal pour cet exercice, la seule consigne est : \og \textit{commencer par chercher dans {\tt /common\_pool/EvaluationNSI/} }\fg.\\
\lpo[2]

\end{document}

